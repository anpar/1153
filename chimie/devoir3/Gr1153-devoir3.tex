\documentclass{article}

\usepackage[utf8]{inputenc}
\usepackage[T1]{fontenc}      
\usepackage[francais]{babel}
\usepackage{chemist}
\usepackage{chemfig} 
\usepackage{lewis}

\newcommand\exercice[1]{%
\paragraph{#1}%
~\par
~\par}

\newcommand\chemfigc[1]{
\vspace{0.5cm}
\begin{center}\chemfig{#1}\end{center}
\vspace{0.5cm}}

\title{Chimie-Physique I - Devoir 3}
\author{Groupe 115.3}
\date{\today}

\begin{document}

\maketitle

\exercice{Exercice 3.22, page 125}

\begin{enumerate}\renewcommand{\theenumi}{\alph{enumi}}
	\item La structure de Lewis du \chemform{CI_4} est :
		
				\chemfigc{\lewis{246, I}-C(-[:90]\lewis{024, I})(-[:-90]\lewis{046, I})-\lewis{026,I}}
				
				La molécule est \textbf{apolaire} (à cause de la symétrie).
	\item	La structure de Lewis du \chemform{CH_3OH} est :
		
				\chemfigc{H-C(-[:90]H)(-[:-90]H)-\lewis{26, O}-H}
				
				La molécule est \textbf{polaire}.
	\item La structure de Lewis du \chemform{CH_3COCH_3} est :	
				
				\chemfigc{H-C(-[:90]H)(-[:-90]H)-C(=[:90]O)-C(-[:90]H)(-[:-90]H)-H}
				
				La molécule est \textbf{polaire}.
\end{enumerate}

\exercice{Exercice 3.100, page 129}

\begin{enumerate}\renewcommand{\theenumi}{\alph{enumi}}
	\item 
				\begin{itemize}
					\item La structure de Lewis du \chemform{H_2CCH_2} est :
		
								\chemfigc{C(-[:135]H)(-[:-135]H)=C(-[:45]H)(-[:-45]H)}
					\item La structure de Lewis du \chemform{H_2CCCH_2} est :
					
								\chemfigc{C(-[:135]H)(-[:-135]H)=C=C(-[:45]H)(-[:-45]H)}
					\item La structure de Lewis du \chemform{H_2CCCCH_2} est :
					
								\chemfigc{C(-[:135]H)(-[:-135]H)=C=C=C(-[:45]H)(-[:-45]H)}
				\end{itemize}
				
	\item	Dans l'ordre, les hybridations pour chaque atomes \chemform{C} sont :
	
				\begin{itemize}
					\item $sp^2$, $sp^2$ ;
					\item $sp^2$, $sp$, $sp^2$ ;
					\item $sp^2$, $sp$, $sp$, $sp^2$.
				\end{itemize}
				
	\item	Dans les trois molécules, les liaisons entre les atomes de carbones sont doubles. On a donc
				une liaison $\sigma$ et une laision $\pi$ entre chaque atome de carbone.
				
				Les liaisons \chemfig{C-H} sont simples, on a donc une liasion $\sigma$.
	\item Les liaisons \chemform{HCH} forment un angle $<$ 120 degrés.
				Les liaisons \chemform{HCC} forment un angle $>$ 120 degrés.
				Les liaisons \chemform{CCC} forment un angle de 180 degrés.
	\item	Oui.
	\item	%TODO.
\end{enumerate}

\end{document}
