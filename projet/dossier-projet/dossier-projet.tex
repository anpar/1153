\documentclass{report}


\usepackage[latin1]{inputenc}
\usepackage[T1]{fontenc}
\usepackage[francais]{babel}
\usepackage{amsmath,amssymb,array}
 
      
\title{Dossier}
\author{Groupe 11.53}
\date{Juin 2014}
\begin{document}
 
\maketitle

Dossier : Séances tutorées.

\section{}
Séance1 : Enoncé du projet : Votre objectif est de réaliser, mesurer et qualifier un dispositif comportant un système d’amplification permettant d’écouter sur deux haut-parleurs de votre fabrication les signaux stéréo provenant de la fiche jack 3,5mm d’un smartphone ou d’un baladeur MP3 et d’en faire varier le volume, l’intensité des sons graves et aigus.
Nous avons réalisé un cahier des charges de notre haut-parleur, ainsi qu’un schéma fonctionnel de l’appareil.  Le cahier des charges comporte les contraintes, les caractéristiques et quelques dimensions de notre haut-parleur.  Le schéma fonctionnel était un plan de toutes les fonctions que devaient avoir pour que le haut-parleur fonctionne.  Cela nous a permis d’avoir une vision globale de la tâche qui nous attendait. Ne connaissant pas grand-chose sur le sujet, nous avons décidé de faire des recherches individuelles sur la fabrication d’un haut-parleur.  

\section{}
Séance 2 : Grâce à des APP en physique qui était surtout axés sur le projet, nous en savons un peu plus sur le fonctionnement d’un tel appareil.  Nous connaissons maintenant les composantes du système et pouvons évaluer les caractéristiques que doivent avoir chaque partie.  Ainsi, nous avons pu calculer le champ magnétique nécessaire pour faire vibrer la membrane.  Une partie du dimensionnement a donc été réalisé assez tôt dans ce projet.  Nous avons également du rechercher une série de mots-clefs en rapport avec le projet.

\section{}
Séance 3 : Nous avons analysé toutes les composantes du circuit et nous avons passé la plupart du temps à comprendre le fonctionnement de la plaquette.  Nous devions savoir comment les différentes parties étaient reliées entre elles.  Comment faire fonctionner la plaquette, avec quelle source de tension,…  Aussi, nous avons commencé  l’analyse mathématique des filtres passe-bas et passe-haut.  Cela n’a été possible qu’après avoir reçu le cours de math de Mr Vitale.  L’approximation mathématique nous a été d’une grande aide dans la résolution de ce problème.

\section{}
Séance 4 : Après avoir compris que nous avions besoins de données claires pour pouvoir avoir l’approximation, nous en avons réalisé au labo.  Avec ces données, nous avons pu les appliquer à la méthode que nous avons choisie pour pouvoir obtenir la fréquence de coupure.  N’obtenant pas la même réponse à chaque fois nous avons dû nous employé pour pouvoir obtenir une réponse correct.  La modélisation du graphe sur un logiciel a permis de prouver que notre estimation était tout à fait correcte.  L’équation différentielle a particulièrement longue et fastidieuse à chercher car le cheminement était long et une erreur est si vite arrivée.

\section{}
Séance 5 : La recherche documentaire devenait une de nos priorités pour le pré-jury qui arrivait.  Nous avons donc été à la bibliothèque par groupe de deux pour y prendre des livres relatant du sujet choisis.  Cela était une première pour la plupart des gens de notre groupe.  Aller sur le site libellule, trouver la référence d’un livre et le consulter de manière efficace pour en trouver l’information n’a pas été une mince affaire mais finalement nous nous trouvions instruit de nouveau termes.  Durant cette séance, nous avons mis en commun les recherches et nous avons finalisé les rapports des deux mots clefs.  

\section{}
Séance 6 : Dernière séance avant le pré-jury donc nous avons répartis le travail qu’il fallait faire pour que tout soit prêt pour le passage devant ce pré-jury.  Nous avons collecté toutes les informations que nous avions pour en faire un rapport amoindri.   Ce qui nous a permis de gagner du temps lors de la rédaction finale du rapport.  Nous avons également approfondis le dimensionnement de des bobines en leur donnant leur taille et caractéristiques finales.  

\section{}
Séance 7 : Pré-jury.  Nous avons défendu notre projet et nous avons vu où était nos erreurs et nos points forts.  

\section{}
Séance 8-9-10 : A partir de cette séance, les séances tutorées ont été moins calculatoire mais plus une préparation en profondeur de ce que nous devions faire en labo.  En effet, presque toute la théorie était réalisée, il fallait savoir pertinemment ce que nous allions faire à la prochaine séance.   Les séances servaient à poser des questions au tuteur et à comprendre ce qui se passait quand quelque chose ne marchait pas au labo.

\end{document}

