\documentclass{article}

\usepackage[utf8]{inputenc}
\usepackage[T1]{fontenc}      
\usepackage[francais]{babel}
\usepackage{graphicx}
\usepackage{circuitikz}
\usepackage[squaren, Gray]{SIunits}
\usepackage{sistyle}
\usepackage[autolanguage]{numprint}
\usepackage{pgfplots}
\usepackage{amsmath,amssymb,array}
\usepackage{url} 

% New command pour la modélisation mécanique, tri à effectuer
\newcommand\fv[1]{{\bf #1}} % free vector
\newcommand\fvd[1]{\dot{\bf #1}} % free vector derivated
\newcommand\fvdd[1]{\ddot{\bf #1}} % free vector derivated
\newcommand\fvr[1]{\mathring{\bf #1}} % free vector relatively derivated
\newcommand\fvrr[1]{\overset{\circ\circ}{\bf #1}} % free vector relatively derivated
\newcommand\uv[1]{{\bf\hat{ #1}}} % unit vector
\newcommand\ui{{\bf\hat{I}}} % unit vector I
\newcommand\uj{{\bf\hat{J}}} % unit vector J
\newcommand\uk{{\bf\hat{K}}} % unit vector K
\newcommand\wrt[2]{\ensuremath{\tensor*[_{ #1}]{ #2}{}}} % With Respect To
\newcommand\wtr[3]{\ensuremath{\tensor*[_{ #1}]{ #2}{^{ #3}}}} % With Two Respect
\newcommand\omegaf{{\bm \omega}}
\newcommand\omegafr{\mathring{\bm \omega}}
\newcommand\omegafd{\dot{\bm \omega}}
\newcommand\omegaft{\tilde{\bm \omega}}
\newcommand\omegaftr{\mathring{\tilde{\bm \omega}}}
\newcommand\omegat{\tilde{\omega}}
\newcommand\omegatd{\tilde{\dot{\omega}}}
\newcommand\ine{{\bf I}}
\newcommand\st{{\bf L}}
\newcommand\pst{{\bf M}}
\newcommand\lm{{\bf N}}
\newcommand\am{{\bf H}}
\newcommand\amd{\dot{\am}}
\newcommand\fo{{\bf F}}
\newcommand\po{\mathcal{P}}
\newcommand\xg{\ensuremath{\fv{R}}}
\newcommand\xgd{\ensuremath{\fvd{R}}}
\newcommand\xgdd{\ensuremath{\fvdd{R}}}
\newcommand\dvec[1]{\dot{\vec{ #1}}}
\newcommand\ddvec[1]{\ddot{\vec{ #1}}}
\newcommand\qp{\dot{q}}
\newcommand\dqp{\Delta \dot{q}}

\begin{document}


\textbf{Groupe \numprint{11.53}}.

\begin{abstract-fr}
% Contexe et tâche
Dans le cadre du cours \textit{Projet 2}, il nous a été demandé
de concevoir un haut-parleur que l'on puisse connecter à un smartphone
par le biais d'une prise Jack.

% Besoin
Pour arriver à nos fins, il a fallu passer par diverses étapes
de modélisations mathématiques et physiques de composants du haut-
parleur. Les situations réelles étant en général trop compliquées à étudier
dans leur globalité (en tout cas à notre stade), ces modélisations se basent sur des 
hypothèses simplificatrices. Malgré que ces hypothèses soient parfois assez fortes, 
elles permettent d'arriver à un modèle relativement cohérent avec les expériences et
mesures effectuées en laboratoire.

% Objet
Ce document décrit en détail chacune des étapes de modélisation effectuées durant
ce projet. Il présente aussi une synthèse des différentes recherches documentaires.

% Résultats, conclusions et perspectives
Bien que notre haut-parleur ne fonctionne pas aussi bien que nous l'aurions
espéré, nous avons énormément appris de ce projet.

Notre état d'esprit est difficile à décrire, nous sommes déçus pour notre haut-parleur
mais à part ça nous sommes satisfaits du travail effectué.

\begin{figure}[!htb]
	\centering
	\includegraphics[scale=0.07]{ext-baffle.jpg}
	\includegraphics[scale=0.07]{int-baffle.jpg}
	\includegraphics[scale=0.35]{circuit.jpg}
	\caption{L'aboutissement du projet : notre haut-parleur. | The result of the project : our loudspeaker.}
\end{figure}

\end{abstract-fr}

\begin{abstract-en}

In this course, \textit{Projet 2}, we have been asked to make a loudspeaker that can be connected to a smartphone with a Jack-plug.

To achieve this task, we had to work out to find a way to go through it.
We had to find mathematical and physical models but the real problem was too tricky for us so we had to make some simpler assumptions.
However, we couldn't make random assumption because the theory had to fit with the test we did in the lab.

In this document, you'll find the necessary calculations and ideas to make such a tool. It also describes some
of our documentary researches.

Even if our loudspeaker doesn't work as well as we wanted, we learned a lot from this challenge. 

Our state of mind is pretty hard to describe : we are disappointed with the actual loudspeaker but quite proud of us for the rest.

\end{abstract-en}

% Just here to fix rapport_prejury.tex
\end{document}