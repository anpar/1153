\documentclass{article}

\usepackage[utf8]{inputenc}
\usepackage[T1]{fontenc}      
\usepackage[francais]{babel}
\usepackage{color}

\title{Errata du rapport}
\author{Groupe 115.3}
\date{\today}

\begin{document}
\maketitle

\section{Fautes de fond}

\begin{itemize}
	\item \textbf{Page 7} : dans le paragraphe intitulé "Recherche de la solution homogène", il est indiqué 
	"$A$ est une constante appartenant à l'ensemble des réels", $A$ appartient en réalité aux complexes ;
	\item \textbf{Page 12} : la perméabilité relative de l'air n'a pas d'unité, or dans le rapport il est mis \unit{}{\henry\per\meter} ;
	\item \textbf{Page 15} : dans la résolution de l'équation différentielle du mouvement, il est noté après le calcul
	des racines du polynôme caractéristiques : "Nous avons donc, en ne gardant que la partie réelle (...)".
	Il faut supprimer la partie "en ne gardant que la partie réelle" ;
	% Trouver l'erreur et reformuler ceci. Et oui on avait bien refait le calcul avec entrefer réduit.
	%\item \textbf{Page 24} : Dans la partie discussion, on dit qu'on a expérimentalement 8 cT, à la place des 7.54 cT calculés, et on explique 
	%cela par le fait qu'un a supposé tout le champ concentré dans l'enterfer... illogique du coup qu'on ait plus que ce qu'on a calculé.
	%Par contre je sais pas si on avait pris en compte le fait qu'on avait rapetissé l'entrefer dans nos calculs, donc ca vient peut
	%etre de là qu'on a un plus grand champ.
	\item \textbf{Page 29} : il est écrit "Pour trouver la fréquence d'intersection entre les deux droites, nous résolvons le système".
	On recherche en fait la fréquence de coupure, qui correspond à l'abscisse à laquelle s'intersecte les
	deux droites, c'est plus précis. De plus, le système a déjà été résolu. Ici on recherche simplement la
	solution de l'équation $-1.96\log{x} + 9.84 = 2.5$. Ce n'est pas vraiment une erreure mais plutôt un manque
	de précision ;
	\item \textbf{Page 30} : il est sans doute plus juste de dire "la méthode utilisant la projection orthogonale comme outil
	de résolution d'un système n'ayant pas de solutions exactes" que de dire "la méthode utilisant les
	bases orthonormées" ;
	% Erreur à vérifier
	%\item \textbf{Page 30} :
	%Je me trompe où cette section parle d'une méthode pour une meilleure approximation mais en parlant
	%d'une autre méthode que cette qu'on a utilisé?
\end{itemize}

\section{Orthographe}
\begin{itemize}
	\item \textbf{Page 12} : "(...) au\textcolor{green}{x} courant\textcolor{green}{s} de \textsc{Foucault} (...)" ;
	\item \textbf{Page 13} : "(...) comme cette bobine mobile est traversé\textcolor{green}{e} par (...)" ;
	\item \textbf{Page 16} : "(...) juste avant qu'elle ne commence\textcolor{red}{r} à retourner (...)" ;
	\item \textbf{Page 22} : "(...) résistan\textcolor{green}{c}e" ;
	\item \textbf{Page 24} : "(...) nous sommes assez fier\textcolor{green}{s}" ;
	\item \textbf{Page 28} : "(...) Soient $f_1$ et \textcolor{red}{ec}$f_2$" ;
\end{itemize}
\end{document}