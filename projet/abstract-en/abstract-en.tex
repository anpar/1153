\documentclass{report}


\usepackage[latin1]{inputenc}
\usepackage[T1]{fontenc}
\usepackage[francais]{babel}
\usepackage{amsmath,amssymb,array}
 
      
\title{abstract}
\author{Groupe 11.53}
\data{2014}
\begin{document}
 
\maketitle

\section{Abstract du haut-parleur}

Le haut-parleur est un outil permettant de transformer un signal électrique en un signal sonore.  Grâce à une bobine fixe, la bobine mobile peut osciller et faire vibrer la membrane à laquelle elle est attachée, ceci crée une onde sonore.
De nombreuses phases de tests ont été effectuées pour obtenir un haut-parleur fonctionnel, malgré les calculs mathématiques nous avons dû faire des hypothèses simplificatrices et donc seuls les vrais tests reflètent la réalité.
Dans ce document, les calculs et les idées nécessaires à a création d’un tel objet sont présentés.  
Notre haut-parleur est capable de produire un son qui a une fréquence entre 500 et 5000 Hz, a une puissance de 2.5 W, avec une membrane de 17 cm de diamètre, une bobine fixe de 450 spires et une mobile de 100.  Avec la prise jack qui permet de relier le haut-parleur à un Gsm ou un mp3, nous pouvons obtenir un son réglable en intensité.

\section{Loud-speaker abstract}

The loud-speaker is a tool that transforms an electric signal into a sound.  Thanks to a fixed coil, a mobile coil can oscillate and because it is attached to a membrane, it creates a sound wave.  
Many tests have been done to get a functional loud-speaker.  We did mathematical calculations but we had to use hypothesis that are simpler than the reality.  So the only way to know if it works or not is to test it.
In this document, calculations and the ideas to could have done such a tool are present.  
Our loud-speaker is able to produce a sound with a frequency between 500 and 5000 Hz and has a power of 2.5 W.  We made the loud-speaker with a membrane of 17 cm of diameter, a fixed coil of 450 spires and mobile one of 100 spires.  With the jack plug ( a cable that can rely a Gsm of a smartphone) we can get a sound and moreover its intensity can be modified.

\end{document}
