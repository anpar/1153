\documentclass{report}

\usepackage[utf8]{inputenc}
\usepackage[T1]{fontenc}      
\usepackage[francais]{babel}
\usepackage{graphicx}
\usepackage{circuitikz}
\usepackage[squaren, Gray]{SIunits}
\usepackage{sistyle}
\usepackage[autolanguage]{numprint}
\usepackage{pgfplots}
\pgfplotsset{compat=1.9}
\usepackage{amsmath,amssymb,array}

\title{Projet 2 - Rapport de Pré-jury}
\author{Groupe 115.3}
\date{\today}

\begin{document}

\maketitle
\tableofcontents
\clearpage

\chapter{Introduction}

Dans le cadre du cours de projet il nous a été demandé de réaliser un haut-parleur, 
que ce soit l'assemblement du circuit électronique, le dimensionnement de l'électroaimant
et de la bobine fixe, ou la membranne. Comme pour le projet du premier quadrimestre, 
celui-ci est intrinséquement lié aux cours donnés lors de ce quadrimestre, à savoir Physique
et Math. Mais plus qu'un cours dispensant des savoirs théoriques, il nous permet de mettre
en pratique ces connaissances théoriques et de développer une méthodologie scientifique, que 
ce soit lors du travail en groupe ou lors de travail individuel.

\chapter{Fonctionnement et dimensionnement}

\documentclass{article}

\usepackage[utf8]{inputenc}
\usepackage[T1]{fontenc}      
\usepackage[francais]{babel}
\usepackage{graphicx}
\usepackage{circuitikz}
\usepackage[squaren, Gray]{SIunits}
\usepackage{sistyle}
\usepackage[autolanguage]{numprint}
\usepackage{pgfplots}
\usepackage{amsmath,amssymb,array}
\usepackage{url} 

% New command pour la modélisation mécanique, tri à effectuer
\newcommand\fv[1]{{\bf #1}} % free vector
\newcommand\fvd[1]{\dot{\bf #1}} % free vector derivated
\newcommand\fvdd[1]{\ddot{\bf #1}} % free vector derivated
\newcommand\fvr[1]{\mathring{\bf #1}} % free vector relatively derivated
\newcommand\fvrr[1]{\overset{\circ\circ}{\bf #1}} % free vector relatively derivated
\newcommand\uv[1]{{\bf\hat{ #1}}} % unit vector
\newcommand\ui{{\bf\hat{I}}} % unit vector I
\newcommand\uj{{\bf\hat{J}}} % unit vector J
\newcommand\uk{{\bf\hat{K}}} % unit vector K
\newcommand\wrt[2]{\ensuremath{\tensor*[_{ #1}]{ #2}{}}} % With Respect To
\newcommand\wtr[3]{\ensuremath{\tensor*[_{ #1}]{ #2}{^{ #3}}}} % With Two Respect
\newcommand\omegaf{{\bm \omega}}
\newcommand\omegafr{\mathring{\bm \omega}}
\newcommand\omegafd{\dot{\bm \omega}}
\newcommand\omegaft{\tilde{\bm \omega}}
\newcommand\omegaftr{\mathring{\tilde{\bm \omega}}}
\newcommand\omegat{\tilde{\omega}}
\newcommand\omegatd{\tilde{\dot{\omega}}}
\newcommand\ine{{\bf I}}
\newcommand\st{{\bf L}}
\newcommand\pst{{\bf M}}
\newcommand\lm{{\bf N}}
\newcommand\am{{\bf H}}
\newcommand\amd{\dot{\am}}
\newcommand\fo{{\bf F}}
\newcommand\po{\mathcal{P}}
\newcommand\xg{\ensuremath{\fv{R}}}
\newcommand\xgd{\ensuremath{\fvd{R}}}
\newcommand\xgdd{\ensuremath{\fvdd{R}}}
\newcommand\dvec[1]{\dot{\vec{ #1}}}
\newcommand\ddvec[1]{\ddot{\vec{ #1}}}
\newcommand\qp{\dot{q}}
\newcommand\dqp{\Delta \dot{q}}

\begin{document}


\section{Dimensionnement du haut-parleur}
Après avoir réalisé quelques recherches sur les haut-parleurs, nous avons pu imaginer le dispositif idéal 
à réaliser. En tenant compte des différentes contraintes qui nous étaient imposées, voici les différents 
choix que nous avons effectués.

\subsection{Pour le boîtier}
Nous devions pouvoir faire varier les fréquences (voir annexe "Cahier des Charges"), ce qui signifie que nous ne 
pouvions pas faire un caisson trop petit. La taille du caisson influence le son restitué par le haut-parleur :
un volume trop petit ne restituerait pas les extrêmes graves. En effet, à de très basses 
fréquences, l'air à  l'interieur du haut-parleur chauffe et la pression augmente.  
L'enceinte close va se comporter comme une raideur supplémentaire qui augmente la fréquence de
résonance\footnote{Fréquence pour laquelle la réponse du circuit est maximale}.  Comme la fréquence de raisonnance se 
situe au milieu de la fréquence de coupure du passe-haut et du passe-bas, si nous augmentons la fréquence de résonnance, 
on augmente la fréquence de coupure du passe-haut\cite{volume} également. Nous avons finalement opté pour un boîtier cubique de 
$\unit{25}{\centi\meter}$ de côté, c'est souvent la forme d'un haut-pareur qu'on peut retrouver dans le commerce.
Nous avions pensé placer un évent à l'avant du haut-parleur pour augmenter le rendement en profitant de l'onde 
arrière, mais c'était plus difficile à construire, et il aurait fallu que l'on accorde l'event, de manière à
exploiter l'onde arrière correctement. Nous nous sommes donc finalement limités à une charge
\footnote{Manière de séparer les ondes avant et arrière.} dite "\textit{close}"\cite{close}.  

Afin d'améliorer un peu le boîtier, nous avons également pensé aux éléments suivants :

\begin{itemize}
	\item	Des pieds en caoutchouc : placer des pieds en caoutchouc sur le boîtier de notre haut-parleur
				permet de réduire les déplacements dûs aux vibrations du haut-parleur ;
	\item	Un bois épais pour le caisson pour empecher le haut-parleur de bouger avec les ondes et éviter que 
	le son produit ne sorte par autre chose que la membrane.
	
\end{itemize}

\subsection{Pour la membrane}
Nous avons opté pour une membrane de diamètre de $\unit{17}{\centi\meter}$. Nous avons choisi cette valeur afin 
d'avoir une membrane assez large, pour exploiter le mieux possible la taille du caisson. C'est également un
diamètre assez répandu dans le commerce\cite{tlhp}. Nous respectons donc les normes.
La profondeur de la membrane est de $\unit{6}{\centi\meter}$, comme pour la plupart des membranes de ce
diamètre\cite{tlhp}. La membrane est réalisée en papier. Pour avoir une constante de raideur nous avons eu besoin
d'un ressort, nous l'avons réalisé en tissus tendu, ce qui est assez inovant pour un haut-parleur.

\paragraph{Tableau récapitulatif}

\begin{table}
	\centering
	\begin{tabularx}{\textwidth}{|X|X|}
		\hline
			 \textbf{Caractéristique} & \textbf{Justification} \\
		\hline
			Volume du caisson : $\unit{25\times25\times25}{\centi\meter}$ & Possibilité de faire varier les fréquences.  \\
		\hline
			Matériau du caisson : Panneau de 	MDF
			d'épaisseur \unit{18}{\milli\meter} & Qualité, robustesse et coût. \\
		\hline
			Diamètre de membrane : \unit{17}{\centi\meter} & Avoir une membrane assez large pour exploiter le mieux possible la taille du caisson. \\
		\hline
			Profondeur de la membrane : \unit{6}{\centi\meter} & Déterminé en fonction du diamètre de la membrane. \\
		\hline
			Materiau membrane : papier et tissus & Rigidité et petite constante de raideur. \\
		\hline
			Masse surfacique du papier : \unit{200}{\gram\per\meter\squared} & Rigidité et coût. \\
		\hline
	\end{tabularx}
\end{table}

% Just here to fix rapport_prejury.tex
\end{document}


\documentclass{article}

\usepackage[utf8]{inputenc}
\usepackage[T1]{fontenc}      
\usepackage[francais]{babel}
\usepackage{graphicx}
\usepackage{circuitikz}
\usepackage[squaren, Gray]{SIunits}
\usepackage{sistyle}
\usepackage[autolanguage]{numprint}
\usepackage{pgfplots}
\usepackage{amsmath,amssymb,array}
\usepackage{url} 

% New command pour la modélisation mécanique, tri à effectuer
\newcommand\fv[1]{{\bf #1}} % free vector
\newcommand\fvd[1]{\dot{\bf #1}} % free vector derivated
\newcommand\fvdd[1]{\ddot{\bf #1}} % free vector derivated
\newcommand\fvr[1]{\mathring{\bf #1}} % free vector relatively derivated
\newcommand\fvrr[1]{\overset{\circ\circ}{\bf #1}} % free vector relatively derivated
\newcommand\uv[1]{{\bf\hat{ #1}}} % unit vector
\newcommand\ui{{\bf\hat{I}}} % unit vector I
\newcommand\uj{{\bf\hat{J}}} % unit vector J
\newcommand\uk{{\bf\hat{K}}} % unit vector K
\newcommand\wrt[2]{\ensuremath{\tensor*[_{ #1}]{ #2}{}}} % With Respect To
\newcommand\wtr[3]{\ensuremath{\tensor*[_{ #1}]{ #2}{^{ #3}}}} % With Two Respect
\newcommand\omegaf{{\bm \omega}}
\newcommand\omegafr{\mathring{\bm \omega}}
\newcommand\omegafd{\dot{\bm \omega}}
\newcommand\omegaft{\tilde{\bm \omega}}
\newcommand\omegaftr{\mathring{\tilde{\bm \omega}}}
\newcommand\omegat{\tilde{\omega}}
\newcommand\omegatd{\tilde{\dot{\omega}}}
\newcommand\ine{{\bf I}}
\newcommand\st{{\bf L}}
\newcommand\pst{{\bf M}}
\newcommand\lm{{\bf N}}
\newcommand\am{{\bf H}}
\newcommand\amd{\dot{\am}}
\newcommand\fo{{\bf F}}
\newcommand\po{\mathcal{P}}
\newcommand\xg{\ensuremath{\fv{R}}}
\newcommand\xgd{\ensuremath{\fvd{R}}}
\newcommand\xgdd{\ensuremath{\fvdd{R}}}
\newcommand\dvec[1]{\dot{\vec{ #1}}}
\newcommand\ddvec[1]{\ddot{\vec{ #1}}}
\newcommand\qp{\dot{q}}
\newcommand\dqp{\Delta \dot{q}}

\begin{document}


\section{Dimensionnement de l'électroaimant et de la bobine mobile}
Pour fabriquer notre haut-parleur, nous ne disposions pas d'aimant permanent. Nous avons donc
dû créer un électroaimant à partir d'un matériau ferromagnétique qui nous a été fourni.
Cette section présente dans un premier temps le dimensionnement de cet électroaimant, c'est-à-dire le
nombre de spires choisi, la résistance totale de la bobine, son inductance, etc.

Nous calculerons ensuite, de manière expérimentale, la constante de raideur de la membrane de
notre haut-parleur. A partir de cela et de l'écartement maximal par rapport à sa position d'origine 
(choisi arbitrairement), 
nous pourons calculer la force nécessaire pour déplacer la membrane, et par conséquent, le nombre
de spires nécessaire sur la bobine mobile.

\subsection{Fonctionnement et dimensionnement de la bobine fixe}
Lorsqu'un courant traverse la bobine de cuivre, un champ magnétique est créé.  Nous obtenons 
donc un électroaimant fixe générant le champ nécessaire au déplacement de la seconde bobine. 
C'est cette seconde bobine qui sera responsable du tremblement de la membrane.

\begin{figure}[ht!]
\centering
\includegraphics[scale=0.6]{electroaimant.png}
\caption{Modélisation d'un électroaimant}
\label{modélisation de l'électroaimant}
\end{figure}

Le nombre de spires de la bobine fixe, appelons-le $N_1$, a été choisi arbitrairement de manière à produire un
champ magnétique assez fort. Nous avons fixé ce nombre, selon les conseils de notre tuteur, à \numprint{400}. 
Nous allons maintenant calculer les caractéristiques suivantes de notre électroaimant :

\begin{itemize}
	\item Résistance totale de la bobine ;
	\item Champ magnétique induit ;
	\item Inductance.
\end{itemize}

% Section à revoir, l'entrefer a changé, le courant qu'on fait passé aussi !
\paragraph{Champ magnétique dans l'entrefer}
Pour céer un champ magnétique plus fort, nous avons réduit l'entrefer de $\unit{x}{\milli\meter}$.
Calculons dans un premier temps le champ magnétique dans l'entrefer de $\unit{x}{\milli\meter}$ en 
utilisant la conservation des flux. Pour ce calcul, nous utilisons l'hypothèse simplificatrice
assez forte que tout le champ se trouve dans l'entrefer.

$$H_e \cdot e = N_1 I \Rightarrow \frac{B_e}{\mu_0 \mu_r} e = N_1 I$$

Pour $N_1 = 400$, l'entrefer $e = \unit{0.011}{\meter}$, $\mu_r = 1.0000004$ la perméabilité magnétique
de l'air et $I = \unit{2.5}{\ampere}$, on trouve alors :

$B_e = \unit{0.1142389}{\tesla}$

\paragraph{Résistance totale de la bobine}
Pour calculer la résistance totale de la bobine, nous devons connaître la longueur totale de fil de cuivre utilisé.
Pour cela nous utilisons la formule suivante :

$$L{fil} = N_1 \cdot 2\pi r$$  

Où $N_1 = 400$ est le nombre de spires de la bobine fixe, et $r$ est la rayon des spires. Pour
$r = \unit{0.016}{\meter}$, on trouve :

$$L_{fil} = \unit{40.3}{\meter}$$

Il ne nous reste donc plus qu'à multiplier la longueur totale trouvée par la résistance linéique des fils de cuivre
($R_{lin} = \unit{0.18}{\ohm\per\meter}$) :

$$R = L_{fil} \cdot R_{lin} = \unit{7.254}{\ohm}$$

\paragraph{Inductance de la bobine}
Une fois le champ magnétique induit connu, l'inductance dans la bobine peut être très facilement calculée par :

$$L = N_1 \frac{\phi_B}{I}

Dans cette formule, il ne nous reste plus qu'à calculer $\phi_B = B \cdot A$ où $A = ab$ est l'aire d'une spire.
On trouve alors :

$L = \unit{0.01475}{\henry}$

\paragraph{Tableau récapitulatif}

\begin{center}
	\begin{tabular}{c|c|c|c|c}
		$N_1$ & $B_e$ & $R$ & $L$ & $L_{fil}$ \\
		\hline
		400 & $\unit{0.1142389}{\tesla}$ & \unit{7.254}{\ohm} & $\unit{0.01475}{\henry}$ & $\unit{40.3}{\meter}$\\
	\end{tabular}
\end{center}

\subsection{Calcul de la constante de raideur de la membrane}
Avant de pouvoir déterminer le nombre de spires de la bobine mouvante, nous avons dû déterminer
expérimentalement la constante de raideur de notre papier pour faire la membrane.
Notre procédure a été la suivante: nous avons suspendu notre membrane, pour ensuite 
déposer un poids dessus, et finalement mesurer l'élongation du matériau.
Nous obtenons ainsi une constante de raideur d'environ \unit {80}{N/m}.

\subsection{Fonctionnement et dimensionnement de la bobine mobile}

\paragraph{Calcul du nombre de spires}
Etant donné que nous disposons d'un amplificateur qui, selon la datasheet, a une puissance de sortie de 
$\unit{2.5}{\watt}$, et que la tension de sortie est de $\unit{15}{\volt}$, nous pouvons trouver le courant
maximal passant dans la bobine mobile:

$$I = \frac{P}{V} = \unit{0.1667}{\ampere}$$

En fonction de la constante de raideur de la membrane trouvée dans la sous-section précédente et de l'écartement
maximal de la membrane par rapport à sa position d'origine (fixé à $d = \unit{0.003}{\meter}$), nous sommes en
mesure de trouver la longueur du fil de la bobine:

$$IL_{fil}B = kx$$
$$L_{fil} = \frac{kx}{IB} = 12.6 m$$

Le fil à notre disposition au laboratoire a un encombrement de $\unit{25.8}{\frac{spires}{cm}}$. Nous otenons 
donc une relation entre $N_2$, le nombre de spires, et $L_{bobine}$, la longueur de la bobine:

$$25.8 = \frac{N_2}{L_{bobine}}$$

En fixant le rayon à \unit{1.7}{mm}, nous pouvons déterminer $N_2$ ainsi que la longueur de la bobine:
$$L_{fil} = N_2 \cdot 2\pi r$$ 
$$N_2 =  \frac{L_{fil}}{2\pi r} = 118$$


\paragraph{Calcul de la résistance totale de la bobine mobile}
Pour calculer la résistance totale de la bobine, il ne nous reste plus qu'à multiplier la longueur de fil trouvée 
précédemment par la résistance linéique du fil de cuivre
($R_{lin} = \unit{0.18}{\ohm\per\meter}$) :

$$R = L_{fil} \cdot R_{lin} = \unit{2.38}{\ohm}$$

\paragraph{Calcul de l'inductance de la bobine mobile}

Une fois le champ magnétique induit connu, l'inductance dans la bobine peut être très facilement calculée par :

$$L = N_2 \frac{\phi_B}{I} = \unit{0.0734}{\henry}$$

\paragraph{Tableau récapitulatif}

\begin{center}
	\begin{tabular}{c|c|c|c}
		$N_2$ & $I$ & $R$ & $L$ \\
		\hline
		 $118$ & $\unit{0.1667}{\ampere}$ & $\unit{2.38}{\ohm}$ & $\unit{0.0734}{\henry}$ \\
	\end{tabular}
\end{center}

\begin{figure}[ht!]
\centering
\includegraphics[scale=0.3]{hautparleur.png}
\caption{Vue d'ensemble avec la seconde bobine}
\label{Vue d'ensemble avec la seconde bobine}
\end{figure}

% Just here to fix rapport_prejury.tex
\end{document}


\chapter{Résultats des problèmes mathématiques}

\documentclass{article}

\usepackage[utf8]{inputenc}
\usepackage[T1]{fontenc}      
\usepackage[francais]{babel}
\usepackage{graphicx}
\usepackage{circuitikz}
\usepackage[squaren, Gray]{SIunits}
\usepackage{sistyle}
\usepackage[autolanguage]{numprint}
\usepackage{pgfplots}
\usepackage{amsmath,amssymb,array}
\usepackage{url} 

% New command pour la modélisation mécanique, tri à effectuer
\newcommand\fv[1]{{\bf #1}} % free vector
\newcommand\fvd[1]{\dot{\bf #1}} % free vector derivated
\newcommand\fvdd[1]{\ddot{\bf #1}} % free vector derivated
\newcommand\fvr[1]{\mathring{\bf #1}} % free vector relatively derivated
\newcommand\fvrr[1]{\overset{\circ\circ}{\bf #1}} % free vector relatively derivated
\newcommand\uv[1]{{\bf\hat{ #1}}} % unit vector
\newcommand\ui{{\bf\hat{I}}} % unit vector I
\newcommand\uj{{\bf\hat{J}}} % unit vector J
\newcommand\uk{{\bf\hat{K}}} % unit vector K
\newcommand\wrt[2]{\ensuremath{\tensor*[_{ #1}]{ #2}{}}} % With Respect To
\newcommand\wtr[3]{\ensuremath{\tensor*[_{ #1}]{ #2}{^{ #3}}}} % With Two Respect
\newcommand\omegaf{{\bm \omega}}
\newcommand\omegafr{\mathring{\bm \omega}}
\newcommand\omegafd{\dot{\bm \omega}}
\newcommand\omegaft{\tilde{\bm \omega}}
\newcommand\omegaftr{\mathring{\tilde{\bm \omega}}}
\newcommand\omegat{\tilde{\omega}}
\newcommand\omegatd{\tilde{\dot{\omega}}}
\newcommand\ine{{\bf I}}
\newcommand\st{{\bf L}}
\newcommand\pst{{\bf M}}
\newcommand\lm{{\bf N}}
\newcommand\am{{\bf H}}
\newcommand\amd{\dot{\am}}
\newcommand\fo{{\bf F}}
\newcommand\po{\mathcal{P}}
\newcommand\xg{\ensuremath{\fv{R}}}
\newcommand\xgd{\ensuremath{\fvd{R}}}
\newcommand\xgdd{\ensuremath{\fvdd{R}}}
\newcommand\dvec[1]{\dot{\vec{ #1}}}
\newcommand\ddvec[1]{\ddot{\vec{ #1}}}
\newcommand\qp{\dot{q}}
\newcommand\dqp{\Delta \dot{q}}

\begin{document}


\section{Modélisation des filtres passe-haut et passe-bas}
Dans cette section, nous allons expliquer la méthode que nous avons
utiliseé pour trouver une expression analytique de la tension de sortie 
dans un filtre passe-bas, la démarche étant la même pour le filtre passe-haut.

Nous avons en réalité utilisé deux méthodes différentes qui, heureusement, 
aboutissent à la même solution. La première méthode utilise ce que nous
avons appris au premier quadrimestre concernant les équations différentielles.
Cette méthode est plus longue et plus compliquée que la deuxième, c'est pourquoi
nous ne la décrirons pas ici.
La deuxième méthode utilise ce que nous avons appris au deuxième quadrimestre 
concernant les équations différentielles et les complexes. 

\subsection{Le filtre passe-bas}

Soit $V_R$ la tension à travers la résistance $R$, $V_C$ la tension à travers
le condensateur $C$, $V_{in}$ la tension d'entrée et $V_{out}$ la tension de
sortie du filtre.

\begin{figure}[ht!]
	\centering
	\begin{circuitikz}
		\draw (0,0) node[ocirc] (A);
		\draw (0,0) to [R=$R$] (2,0);
		\draw (2,0) to [short] (4,0);
		\draw (4,0) node[ocirc] (C);
		\draw (2,0) to [C=$C$] (2,-2);
		\draw (2,-2) to [short] (4,-2);
		\draw (4,-2) node[ocirc] (D);
		\draw (0,-2) to [short] (2,-2);
		\draw (0,-2) node[ocirc] (B);
		\draw (A) to[open, v=$V_ {in}$] (B);
		\draw (C) to[open, v=$V_{out}$] (D);
	\end{circuitikz}
	\caption{Schéma électrique d'un filtre passe-bas}
	\label{lwp_scheme}
\end{figure}

Sur le circuit ci-dessus (Figure \ref{lwp_scheme}), on peut utiliser la loi des tensions de Kirchhoff :

$$V_{in} = V_R + V_{out}$$

On note $V$ l'amplitude de la tension d'entrée sinusoïdale, $i(t)$ est le courant
en fonction du temps : 

$$V \cdot \cos (\omega t) = R \cdot i(t) + V_C$$

Or, le courant $i(t)$ à travers un condensateur est donné par $C \frac{dV_C}{dt}$, 
l'équation devient alors une équation différentielle en la fonction inconnue $V_C (t)$ :

$$V \cdot \cos (\omega t) = RC\frac{dV_C}{dt}  + V_C$$

On peut réecrire cette équation de la manière suivante, où $y = V_C(t)$ :

$$RCy' + y = V \cdot \cos (\omega t)$$

Cette équation va être la base de la méthode qui suit. On va également utiliser 
la condition initiale suivante :

$$y(0) = 0$$

\subsubsection{Résolution de l'équation différentielle}

On sait que $\cos (\omega t)$ est égale à la partie réelle de l'exponentielle
complexe $e^{\omega i t}$. On réecrit alors l'équation différentielle de la
manière suivante :

$$RCy' + y = V \cdot e^{\omega i t}$$

Comme pour toute équation différentielle linéaire non-homogène, nous allons travailler
en deux étapes :

\paragraph{Recherche de la solution homogène}

Le polynôme caractéristique de l'équation homogène est :

$$RC \cdot x + 1 = 0$$

On a alors $x = \frac{-1}{RC}$ comme racine, et on trouve donc comme solution homogène :

$$y_h(t) = A \cdot e^{\frac{-t}{RC}}$$

Où $A$ est une constante appartenant à l'ensemble des réels. % A confirmer, j'ai un doute.

\paragraph{Recherche de la solution particulière}

La solution particulière qu'on recherche est de la forme :

$$y_p(t) = \alpha \cdot e^{\omega i t}$$

Il nous reste donc à déterminer la constante complexe $\alpha$. Pour ce faire,
nous injectons dans l'équation de départ $y_p(t)$ et sa dérivée première. On
trouve alors :

$$\alpha = \frac{V(1-RC\omega i)}{1+R^2C^2\omega^2}$$

La solution particulière est donc :

$$y_p(t) = \frac{V(1-RC\omega i)}{1+R^2C^2\omega^2} \cdot e^{\omega i t}$$

\paragraph{Solution complète}

La solution finale $y(t)$ est égale à $y_h(t) + y_p(t)$ :

$$y(t) = A \cdot e^{\frac{-t}{RC}} + \frac{V(1-RC\omega i)}{1+R^2C^2\omega^2} \cdot e^{\omega i t}$$

En retransformant ensuite l'exponentielle complexe en sa forme trigonométrique et en ne
gardant que la partie réelle, on trouve :

$$y(t) = V_C(t) = \frac{V(\cos (\omega t) + RC\omega \sin (\omega t))}{1 + \omega^2R^2C^2} + A \cdot e^{\frac{-t}{RC}}$$

\paragraph{Elimination de la constante}

Il ne nous reste plus qu'à éliminer la constante $A$ en utilisant la condition initiale.
On trouve enfin :

$$A = -\frac{V}{1 + \omega^2R^2C^2}$$                         

\paragraph{Conclusion}

La tension de sortie en fonction du temps est donc donnée par :

$$V_{out} = \frac{V}{1 + \omega^2R^2C^2} \cdot (\cos (\omega t) + RC\omega \sin (\omega t) - e^{\frac{-t}{RC}})$$

On peut ensuite réecrire cette formule de manière à faire apparaître
le déphasage de la tension de sortie par rapport à la tension d'entrée. En transformant
$y_p(t)$ en utilisant la notation exponentielle $|z|e^{\phi i}$ d'un couple de la forme 
$a+bi$ et en utilisant ensuite la notation trigonométrique d'une exponentielle complexe,
on trouve, après quelques simplifications et mises en évidence :

$$V_{out} = \frac{V}{\sqrt{1 + R^2\omega^2C^2}}
\left(-\frac{e^{\frac{-t}{RC}}}{\sqrt{1 + R^2\omega^2C^2}} + \cos(\arctan(-RC\omega) + \omega t)\right)$$

On remarque donc que le déphasage entre $V_{out}$ et $V_{in}$ est $-\arctan(RC\omega) = -\arctan(2\pi fRC)$.
Ce déphasage augmente donc linéairement avec $\omega$ et est dû au temps que met le condensateur
à se charger. % A vérifier

\subsubsection{Vérification des résultats}

Une première vérification que l'on peut faire est de vérifier que $V_{out}$ tend vers 0
lorsque $\omega$ tend vers l'infini. C'est bien le cas ici puisqu'on a $\omega^2$ au dénominateur.

On peut ensuite regarder les graphes de $V_{out}$, $V_{in}$ (Figure \ref{lwp_voltages}) et $V_{out} / V_{in}$
(Figure \ref{lwp_ratio}).

\begin{figure}[ht!]
	\centering
	\begin{tikzpicture}[>=stealth]
    \begin{axis}[
        xmin=0,xmax=6,
        ymin=-8,ymax=8,
        axis x line=middle,
        axis y line=middle,
        axis line style=->,
        xlabel={$V$},
        ylabel={$t$},
        ]
				
        \addplot[no marks,black,-] expression[domain=0:6,samples=1000]
						{((7.5)/(sqrt(1 + 1000^2 * 0.00001^2 * 400^2))) * (((-2.718^((-x)/(1000*0.00001)))/(sqrt(1 + 1000^2 * 0.00001^2 * 400^2))) 
						+ cos(atan(-1000*0.00001*400) + 400*x))} 
						node[pos=0.65,anchor=south west]{$$};
						
				\addplot[no marks,blue,-] expression[domain=0:6,samples=1000]
						{7.5 * cos(400 * x)} 
						node[pos=0.65,anchor=south west]{$$}; 

    \end{axis}
	\end{tikzpicture}
	\caption{Graphe de $V_{out}$ (en noir) et $V_{in}$ (en bleu) pour les valeurs suivantes : $V_{max} = \unit{7.5}{\volt}$, $C = \unit{0.00001}{\farad}$,
						$R = \unit{1000}{\ohm}$ et $f = \unit{63.66}{\hertz}$}
	\label{lwp_voltages}
\end{figure}

\begin{figure}[ht!]
	\centering
	\begin{tikzpicture}[>=stealth]
    \begin{axis}[
        xmin=0,xmax=1400,
        ymin=0,ymax=1.2,
        axis x line=middle,
        axis y line=middle,
        axis line style=->,
        xlabel={$f$},
        ylabel={$V_{out} / V_{in}$},
        ]
				
				\addplot[no marks,green,-] expression[domain=0:1400,samples=100]
						% Formule par rapport aux expressions obtenues, un peu décallée
						% {(((7.5)/(sqrt(1 + 100^2 * 0.00001^2 * (2*3.14*x)^2))) * (((-2.718^((-100*0.00001)/(100*0.00001)))/(sqrt(1 + 100^2 * 
						% 0.00001^2 * (2*3.14*x)^2))) + cos(atan(-100*0.00001*2*3.14*x) + 2*3.14*x*100*0.00001)))/(7.5 * cos(2*3.14*x*100*0.00001))}
						{(1 + (2*3.14*x*100*0.00001)^2)^(-0.5)}
						node[pos=0.65,anchor=south west]{$$}; 
    \end{axis}
	\end{tikzpicture}
	\caption{Graphe de $V_{out} / V_{in}$ pour les valeurs suivantes : $R = \unit{100}{\ohm}$ et $C = {\unit{0.0001}{\farad}}$.}
	\label{lwp_ratio}
\end{figure}

\bigbreak

\subsection{Le filtre passe-haut}

Soit $V_R$ la tension à travers la résistance $R$, $V_C$ la tension à travers
le condensateur $C$, $V_{in}$ la tension d'entrée et $V_{out}$ la tension de
sortie du filtre.

\begin{figure}[ht!]
	\centering
	\begin{circuitikz}
		\draw (0,0) node[ocirc] (A);
		\draw (0,0) to [C=$C$] (2,0);
		\draw (2,0) to [short] (4,0);
		\draw (4,0) node[ocirc] (C);
		\draw (2,0) to [R=$R$] (2,-2);
		\draw (2,-2) to [short] (4,-2);
		\draw (4,-2) node[ocirc] (D);
		\draw (0,-2) to [short] (2,-2);
		\draw (0,-2) node[ocirc] (B);
		\draw (A) to[open, v=$V_ {in}$] (B);
		\draw (C) to[open, v=$V_{out}$] (D);
	\end{circuitikz}
	\caption{Schéma électrique d'un filtre passe-haut.}
	\label{hgp_scheme}
\end{figure}

Sur la Figure \ref{hgp_scheme}, la loi des tensions de Kirchhoff donne la même équation que pour le filtre passe-bas :

$$V_{in} = V_R + V_C$$

Cette fois, $V_{out} = V_R$. Or on connait déjà $V_C$ que l'on a calculé dans
le section précédente. On a alors simplement :

$$V_R = V_{in} - V_C$$

$$V_{out} = \frac{V}{\sqrt{1 + R^2\omega^2C^2}}
\left(\frac{e^{\frac{-t}{RC}}}{\sqrt{1 + R^2\omega^2C^2}} - \cos(\arctan(-RC\omega) + \omega t) \right) + \cos(\omega t)$$


Le déphasage reste donc le même que pour le filtre passe-bas.

\subsubsection{Vérification des résultats}

Pour le filtre passe-haut, on va cette fois vérifier que lorsque $\omega$ tend vers 0, on a
$V_{out}$ qui tend vers 0 également. Une fois de plus, c'est bien le cas.

On peut ensuite comparer les graphes de $V_{out}$, $V_{in}$ (Figure \ref{hgp_voltages}) et $V_{out} / V_{in}$
(Figure \ref{hgp_ratio}).

\begin{figure}[ht!]
	\centering
	\begin{tikzpicture}[>=stealth]
    \begin{axis}[
        xmin=0,xmax=6,
        ymin=-8,ymax=8,
        axis x line=middle,
        axis y line=middle,
        axis line style=->,
        xlabel={$V$},
        ylabel={$t$},
        ]
				
        \addplot[no marks,black,-] expression[domain=0:6,samples=1000]
						{(7.5 * cos(100*x)) - ((7.5)/(sqrt(1 + 1000^2 * 0.00001^2 * 100^2))) * (((-2.718^((-x)/(1000*0.00001)))/(sqrt(1 + 1000^2 *
						0.00001^2 * 100^2))) + cos(atan(-1000*0.00001*100) + 100*x))} 
						node[pos=0.65,anchor=south west]{$V_{out}$};
						
				\addplot[no marks,blue,-] expression[domain=0:25,samples=1000]
						{7.5 * cos(100 * x)} 
						node[pos=0.65,anchor=south west]{$V_{in}$}; 
				
    \end{axis}
	\end{tikzpicture}
	\caption{Graphe de $V_{out}$ et $V_{in}$ pour les valeurs suivantes : $V_{max} = \unit{7.5}{\volt}$, $C = \unit{0.00001}{\farad}$,
					$R = \unit{1000}{\ohm}$ et $f = \unit{15.91}{\hertz}$}
	\label{hgp_voltages}
\end{figure}

\begin{figure}[ht!]
	\centering
	\begin{tikzpicture}[>=stealth]
    \begin{axis}[
        xmin=0,xmax=1400,
        ymin=0,ymax=1,
        axis x line=middle,
        axis y line=middle,
        axis line style=->,
        xlabel={$f$},
        ylabel={$V_{out}/V_{in}$},
        ]

				\addplot[no marks,green,-] expression[domain=0:1400,samples=100]
				% Formule obtenue avec nos expressions, décallée de 0.4 vers le haut.
				%		{((7.5 * cos(2*3.14*x*100*0.00001)) - ((7.5)/(sqrt(1 + 100^2 * 0.00001^2 * (2*3.14*x)^2))) * 		
				%	(((-2.718^((-100*0.00001)/(100*0.00001)))/(sqrt(1 + 100^2 *0.00001^2 * (2*3.14*x)^2))) + cos(atan(-100*0.00001*2*3.14*x) +
				%	2*3.14*x*100*0.00001)))/(7.5 * cos(2*3.14*x*100*0.00001))}
				{(1 + (1)/((2*3.14*x*100*0.00001)^2))^(-0.5)}
						node[pos=0.65,anchor=south west]{$$}; 

    \end{axis}
	\end{tikzpicture}
	\caption{Graphe de $V_{out} / V_{in}$ pour les valeurs suivantes : $R = \unit{100}{\ohm}$ et $C = {\unit{0.0001}{\farad}}$.}
	\label{hgp_ratio}
\end{figure}

\subsection{Le filtre passe-bande}

Le filtre passe-bande sert, comme son nom l'indique, à laisser passer une certaine
bande de fréquences. Il est constitué d'un filtre passe-haut suivi d'un passe-bas, 
ou inversément. Les fréquences de coupure respectives des filtres déterminent 
l'ampleur de la bande passante. Plus la résistance pour le filtre passe-bas 
(resp.passe-haut) est petite (resp.grande), plus la bande passante est large, 
étant donné que la fréquence de coupure est inversément proportionnelle à la 
résistance. Nous nous intéresserons ici à un signal passant d'abord par un filtre 
passe-haut, et ensuite par le filtre passe-bas.

Soit $V_{in1}$ la tension à l'entrée du filtre passe-bas, $R_{1}$ la résistance, 
et $C_{1}$ la capacité. Dans la section précédente, nous sommes arrivés au résultat suivant:

$$V_{out1} = \frac{V_{in1}}{\sqrt{1 + R_{1}^2\omega^2C_{1}^2}}
\left (-\frac{e^{\frac{-t}{R_{1}C_{1}}}}{\sqrt{1 + R_{1}^2\omega^2C_{1}^2}} + 
\cos(\arctan(-R_{1}C_{1}\omega) + \omega t)\right)$$

Cette tension de sortie du filtre passe-bas sera notre tension d'entrée pour le 
filtre passe-haut. Précédemment, dans la section concernant le filtre passe haut,
nous trouvions:

$$V_{out2} = \frac{V_{in2}}{\sqrt{1 + R_{2}^2\omega^2C_{2}^2}}
\left(\frac{e^{\frac{-t}{R_{2}C_{2}}}}{\sqrt{1 + R_{2}^2\omega^2C_{2}^2}} - 
\cos(\arctan(-R_{2}C_{2}\omega) + \omega t)\right) + V_{in2}\cos(\omega t)$$

où $V_{in2}$ est la tension à l'entrée du filtre passe-haut, $R_{2}$ la résistance, 
et $C_{2}$ la capacité. 
En remplaçant $V_{in2}$ par $V_{out1}$, la tension à la sortie du passe-bas, nous 
trouverons $V_{out3}$, la tension de sortie finale.
Après simplification, nous obtenons:


$$V_{out3} = \frac{V_{out1} \cdot V_{out2}}{V_{in1}}$$

\begin{figure}[ht!]
\centering
\begin{tikzpicture}[>=stealth]
\begin{axis}[
xmin=0,xmax=50000,
ymin=0,ymax=0.002,
axis x line=middle,
axis y line=middle,
axis line style=->,
xlabel={$f$},
ylabel={$V_{out}/V_{in}$},]
\addplot[no marks,red,-] expression[domain=0:100000,samples=500]
% Formule par rapport aux expressions obtenues, un peu décallée
% {(((7.5)/(sqrt(1 + 100^2 * 0.00001^2 * (2*3.14*x)^2))) * (((-2.718^((-100*0.00001)/(100*0.00001)))/(sqrt(1 + 100^2 *
% 0.00001^2 * (2*3.14*x)^2))) + cos(atan(-100*0.00001*2*3.14*x) + 2*3.14*x*100*0.00001)))/(7.5 * cos(2*3.14*x*100*0.00001))}
{ (((1 + (2*3.14*x*1000*0.000000470)^2)^(-0.5))*((1 + (1)/((2*3.14*x*10*0.000000470)^2))^(-0.5))/7.5)}
node[pos=0.65,anchor=south west]{$$};
\end{axis}
\end{tikzpicture}
\caption{Graphe de $V_{out3} / V_{in1}$ pour le passe-bande}
\label{hgp_ratio}
\end{figure}


\subsubsection{Vérification des résultats}

Au vu du graphe de $V_{out3} / V_{in1}$ de l'équation obtenue pour le passe-bande, nous pouvons
valider notre résultat, étant donné que l'allure du graphique correspond à nos attentes. En effet,
nous pouvons apercevoir très nettement une première fréquence de coupure, et en envisager une autre.

% Just here to fix rapport_prejury.tex
\end{document}


\documentclass{article}

\usepackage[utf8]{inputenc}
\usepackage[T1]{fontenc}
\usepackage[francais]{babel}
\usepackage{amsmath,amssymb,array}
 
\begin{document}
 
\section{Approximation de la fréquence de coupure}

\subsection{Pour le filtre passe-bas}

\subsubsection{Equation de la droite horizontale} % A refaire avec la méthode d'approximation
Nous savons que la droite horizontale a une valeur initiale de 2.5 V et donc \[y=2.5\]

\subsubsection{Equation de la droite diagonale}

Nous savons que l'équation de la droite est de type $y=a*x+b$
\\
Mais pour cette situation-ci, nous allons utiliser une base logarithmique pour la pente.  En effet, les différentes fréquences utilisées sont tellement éloignées les unes des autres que le graphique serait gigantesque et la pente diagonale serait en fait une courbe.  Ce qui donne: $y=a*\log{x}+b$

Voici 3 résultats mesurés en laboratoire.

\bigbreak
\\
\begin{tabular}{|c|c|c|}
\hline
V_c & f & log{ f} \\
\hline
1.7 & 16000 & 4.204\\
\hline
1.55 & 18000 & 4.255\\
\hline
1.45 & 20000 & 4.301 \\
\hline
\end{tabular}

\bigbreak
Des maintenant les fréquences sont exprimées en base logarithmique et nous obtenons la matrice suivante :
\bigbreak

$$
\begin{pmatrix}  
 4.204 & 1\\
 4.255 & 1 \\
 4.301 & 1 
\end{pmatrix}
\begin{pmatrix}  
a\\
b
\end{pmatrix}
=
\begin{pmatrix}  
1.7\\
1.55\\
1.45
\end{pmatrix}
$$

\bigbreak

Ce qui nous donne les vecteurs suivants:

\[e_1=( \frac{1}{\sqrt[]{3}} \frac{1}{\sqrt[]{3}} \frac{1}{\sqrt[]{3}})\]

\\ et

\\
\[e_2=( -0.68, 0.03, 0.73)\]

\bigbreak
Ce qui nous donne une projection de 
$$
\begin{pmatrix}  
1.6\\
1.5\\
1.4
\end{pmatrix}$$
$$

\bigbreak
Nous en déduisons la valeur des coefficients a et b:  
\[ a =-1.96 \]
\[ b= 9.84 \]

$$\fbox{y= -1.96 \timeslog{x} +9.84}$$

\bigbreak
Pour trouver la fréquence d'intersection entre les deux droites $$y=2.5$$ et $$y= -1.96 \times log{x} +9.84$$ nous égalisons les y et nous trouvons $$\fbox{x=5557.7 Hz$$} 

\\
Cela nous semble correct car en théorie nous devons arriver à une valeur f tel que $$f=\frac{1}{2\times \pi\times R\times C}$$
avec $R=7.5+50=57.5 ohms$ et $C=470\times 10^{-9} F$  Notre valeur théorique de la fréquence est donc $$f=5889.2 Hz$$


\subsection{Pour le filtre passe-haut}

\subsubection{Equation de la droite horizontale}

Nous savons que la droite a une valeur initiale de 0.75 V et donc \[y=0.75\]

\subsubsection{Equation de la droite diagonale}

Nous savons que l'équation de la droite est de type $y=a*x+b$
\\
Mais pour cette situation-ci, nous allons utiliser une base logarithmique pour la pente.  En effet, les différentes fréquences utilisées sont tellement éloignées les unes des autres que le graphique serait gigantesque et la pente diagonale serait en fait une courbe.  Ce qui donne: $y=a*\log{x}+b$


Voici 3 résultats mesurés en laboratoire.
\bigbreak
\\
\begin{tabular}{|c|c|c|}
\hline
V_c & f & log{ f} \\
\hline
127 & 0.4 & 2.1\\
\hline
191 & 0.5 & 2.3\\
\hline
356 & 0.6 & 2.6 \\
\hline
\end{tabular}

\bigbreak
Des maintenant les fréquences sont exprimées en base logarithmique et nous obtenons la matrice suivante:
\bigbreak
$$
\begin{pmatrix}  
 2.1 & 1\\
 2.3 & 1 \\
 2.6 & 1 
\end{pmatrix}
\begin{pmatrix}  
a\\
b
\end{pmatrix}
=
\begin{pmatrix}  
0.4\\
0.5\\
0.6
\end{pmatrix}
$$
\bigbreak

Ce qui nous donne les vecteurs suivants:

\[e_1=( \frac{1}{\sqrt[]{3}} \frac{1}{\sqrt[]{3}} \frac{1}{\sqrt[]{3}})\]

\\ et

\\
\[e_2=( -0.6, 0, 0.8)\]

\bigbreak
Ce qui nous donne une projection de 
$$
\begin{pmatrix}  
0.36\\
0.5\\
0.69
\end{pmatrix}$$
$$

\bigbreak
Nous en déduisons la valeur des coefficients a et b:  
\[ a =0.7 \]
\[ b= -1.11 \]

$$\fbox{y= -1.96 \timeslog{x} +9.84}$$

\bigbreak
Pour trouver la fréquence d'intersection entre les deux droites $$y=0.75$$ et $$y= 0.7 \times log{x} -1.11$$ nous égalisons les y et nous trouvons $$\fbox{x=439.4 Hz$$} 

\end{document}


\chapter{Recherches documentaires}

\documentclass{article}

\usepackage[utf8]{inputenc}
\usepackage[T1]{fontenc}      
\usepackage[francais]{babel}
\usepackage{graphicx}
\usepackage{circuitikz}
\usepackage[squaren, Gray]{SIunits}
\usepackage{sistyle}
\usepackage[autolanguage]{numprint}
\usepackage{pgfplots}
\usepackage{amsmath,amssymb,array}
\usepackage{url} 

% New command pour la modélisation mécanique, tri à effectuer
\newcommand\fv[1]{{\bf #1}} % free vector
\newcommand\fvd[1]{\dot{\bf #1}} % free vector derivated
\newcommand\fvdd[1]{\ddot{\bf #1}} % free vector derivated
\newcommand\fvr[1]{\mathring{\bf #1}} % free vector relatively derivated
\newcommand\fvrr[1]{\overset{\circ\circ}{\bf #1}} % free vector relatively derivated
\newcommand\uv[1]{{\bf\hat{ #1}}} % unit vector
\newcommand\ui{{\bf\hat{I}}} % unit vector I
\newcommand\uj{{\bf\hat{J}}} % unit vector J
\newcommand\uk{{\bf\hat{K}}} % unit vector K
\newcommand\wrt[2]{\ensuremath{\tensor*[_{ #1}]{ #2}{}}} % With Respect To
\newcommand\wtr[3]{\ensuremath{\tensor*[_{ #1}]{ #2}{^{ #3}}}} % With Two Respect
\newcommand\omegaf{{\bm \omega}}
\newcommand\omegafr{\mathring{\bm \omega}}
\newcommand\omegafd{\dot{\bm \omega}}
\newcommand\omegaft{\tilde{\bm \omega}}
\newcommand\omegaftr{\mathring{\tilde{\bm \omega}}}
\newcommand\omegat{\tilde{\omega}}
\newcommand\omegatd{\tilde{\dot{\omega}}}
\newcommand\ine{{\bf I}}
\newcommand\st{{\bf L}}
\newcommand\pst{{\bf M}}
\newcommand\lm{{\bf N}}
\newcommand\am{{\bf H}}
\newcommand\amd{\dot{\am}}
\newcommand\fo{{\bf F}}
\newcommand\po{\mathcal{P}}
\newcommand\xg{\ensuremath{\fv{R}}}
\newcommand\xgd{\ensuremath{\fvd{R}}}
\newcommand\xgdd{\ensuremath{\fvdd{R}}}
\newcommand\dvec[1]{\dot{\vec{ #1}}}
\newcommand\ddvec[1]{\ddot{\vec{ #1}}}
\newcommand\qp{\dot{q}}
\newcommand\dqp{\Delta \dot{q}}

\begin{document}


\section{La contre-réaction ou réaction négative}
En analysant le circuit de notre haut-parleur, nous avons découvert la présence de boucles reliant 
la sortie et la borne négative des amplificateurs. Nous nous sommes alors interrogés sur le rôle de ces boucles.

Nous allons dans un premier temps expliquer les raisons d'être des boucles de contre-réaction en général et 
nous finirons par l'explication complète de leur raison d'être dans le cas particulier de notre circuit.

\subsection{Principe de la réaction}
Le principe de la réaction est présent dans un grand nombre de circuits électroniques. Il consiste en une 
réinjection d'une partie du signal de sortie à l'entrée du circuit pour le combiner avec le signal d'entrée 
extérieur\cite{correvon}.

Il existe deux types de réactions\cite{correvon} :

\begin{itemize}
	\item \textbf{La réaction positive} : le signal réinjecté est en phase avec le signal d'entrée de telle 
	sorte que les deux signaux s'additionnent ;
	\item \textbf{La réaction négative} (ou contre-réaction) : le signal réinjecté est en opposition de 
	phase avec le signal d'entrée, de telle sorte que les deux signaux
	se soustraient.
\end{itemize}

\begin{figure}[!htb]
	\centering
	\begin{circuitikz}
		\draw (0, 0) node[ocirc]{};
		\draw (0, 0)	to[short] (2, 0);
		\draw (0, -1) node[ocirc]{};
		\draw (0, -1) to[short] (2, -1);
		\draw (3.1, -0.5) node [op amp, yscale=-1.022] (op amp) {}
					(opamp.-)node[left]{}
					(opamp.+)node[left]{}
					(opamp.out)node[right]{};
		\draw (3.85, -0.5) to[short] (5.6, -0.5);
		\draw (5.6, -0.5) node[ocirc]{};
		\draw (5.4, -0.5) to[short] (5.4, -2);
		\draw (5.4, -2) to[short] (1.4, -2);
		\draw (1.4, -2) to[short] (1.4, -1);
	\end{circuitikz}
	\caption{Schéma électrique d'une boucle de réaction sur un 	amplificateur.}
	\label{reaction1}
\end{figure}

\subsection{Effets des boucles de contre-réaction}

\subsubsection{En général}
Les effets des boucles de contre-réaction sur un amplificateur sont nombreux\cite{sporken}\cite{dusausay} :

\begin{itemize}
	\item La boucle de contre-réaction rend indépendant le gain de l'amplificateur des différentes variations du circuit\cite{lynch} ;
	\item Le signal de sortie est plus proche du signal d'entrée que si l'amplificateur avait été en boucle ouverte ;
	\item Réduction des signaux électriques parasites et de la distorsion dûs à l'amplificateur : en boucle ouverte, 
	le taux de distorsion d'un amplificateur est typiquement de 1\%. La boucle de contre-réaction permet de diminuer ce taux à 0.001\% ;
	\item Contrôle du gain de l'amplificateur (qui est, en boucle ouverte, de l'ordre de $10^6$) ;
	\item Élargissement de la bande passante de l'amplificateur ;
	\item Réduction de l'impédance de sortie.
\end{itemize}

\subsubsection{Intégration dans le circuit du haut-parleur}
Dans notre cas particulier, le principal effet de la boucle de contre-réaction est le contrôle du gain de l'amplificateur 
qui ramène à $1$ le gain.

\begin{figure}[!htb]
	\centering
	\begin{circuitikz}
		\draw (0,0) node[ocirc]{};
		\draw (3,0) to[short] (opamp+);
		\draw (4, -0.5) node [op amp, yscale=-1.022] (op amp){}
			(opamp.-)node[left] (opamp-){}
			(opamp.+)node[left] (opamp+){}
			(opamp.out)node[right] (opampout){};
		\draw (5, -0.5) to[short] (7, -0.5);
		\draw (7, -0.5) node[ocirc]{};
		\draw (2, -1) to[short] (3, -1);
		\draw (2, -1) to[short] (2, -3);
		\draw (2, -3) to[R=$R_1$] (2, -4);
		\draw (2, -4) to[short] (2, -4.5);
		\draw (2, -4) node[ground]{};
		\draw (2, -2) to[short] (6, -2);
		\draw (6, -2) to[R=$R_2$] (6, -0.5);
	\end{circuitikz}
	\caption{Schéma électrique d'une boucle de réaction sur un 	amplificateur avec un diviseur résistif.}
	\label{reaction2}
\end{figure}

Sur la Figure \ref{reaction2}, nous remarquons que la tension de sortie et la tension d'entrée sont liées 
par la formule des diviseurs résistifs :

$$V_{in} = \frac{R_1}{R_1 + R_2} V_{out}$$

Le gain est alors donné par :

$$A = \frac{V_{out}}{V_{in}} = \frac{R_1 + R_2}{R_1}$$

Pour réduire le gain $A$ à $1$, deux possibilités s'offrent à nous:

\begin{enumerate}
	\item	Choisir $R_1 >> R_2$ ;
	\item Choisir $R_2 = 0$ ;
\end{enumerate}

La possibilité la plus simple est la deuxième, car en choississant $R_2 = 0$, le gain est donné par $\frac{R_1}{R_1}$. 
Autrement dit : quelque soit $R_1$, on a $A = 1$ de telle sorte que $V_{in} = V_{out}$. On choisit alors $R_1$ si petit 
que le remplacer par un simple court-circuit a le même effet.

Dans un tel montage (appelé \textit{suiveur de tension}), la résistance d'entrée est infinie alors que la résistance de 
sortie est faible. Le courant de sortie est alors plus grand que le courant d'entrée (qui est presque nul).

Dans notre circuit, ces suiveurs de tension ont un rôle important puisqu'ils permettent le règlage indépendant des
graves et des aigus. Sans eux, modifier la résitance dans le filtre passe-bas modifierait aussi la résistance dans
le filtre passe-haut.

% Just here to fix rapport_prejury.tex
\end{document}


\documentclass{article}

\usepackage[utf8]{inputenc}
\usepackage[T1]{fontenc}      
\usepackage[francais]{babel}
\usepackage{graphicx}
\usepackage{circuitikz}
\usepackage[squaren, Gray]{SIunits}
\usepackage{sistyle}
\usepackage[autolanguage]{numprint}
\usepackage{pgfplots}
\usepackage{amsmath,amssymb,array}
\usepackage{url} 

% New command pour la modélisation mécanique, tri à effectuer
\newcommand\fv[1]{{\bf #1}} % free vector
\newcommand\fvd[1]{\dot{\bf #1}} % free vector derivated
\newcommand\fvdd[1]{\ddot{\bf #1}} % free vector derivated
\newcommand\fvr[1]{\mathring{\bf #1}} % free vector relatively derivated
\newcommand\fvrr[1]{\overset{\circ\circ}{\bf #1}} % free vector relatively derivated
\newcommand\uv[1]{{\bf\hat{ #1}}} % unit vector
\newcommand\ui{{\bf\hat{I}}} % unit vector I
\newcommand\uj{{\bf\hat{J}}} % unit vector J
\newcommand\uk{{\bf\hat{K}}} % unit vector K
\newcommand\wrt[2]{\ensuremath{\tensor*[_{ #1}]{ #2}{}}} % With Respect To
\newcommand\wtr[3]{\ensuremath{\tensor*[_{ #1}]{ #2}{^{ #3}}}} % With Two Respect
\newcommand\omegaf{{\bm \omega}}
\newcommand\omegafr{\mathring{\bm \omega}}
\newcommand\omegafd{\dot{\bm \omega}}
\newcommand\omegaft{\tilde{\bm \omega}}
\newcommand\omegaftr{\mathring{\tilde{\bm \omega}}}
\newcommand\omegat{\tilde{\omega}}
\newcommand\omegatd{\tilde{\dot{\omega}}}
\newcommand\ine{{\bf I}}
\newcommand\st{{\bf L}}
\newcommand\pst{{\bf M}}
\newcommand\lm{{\bf N}}
\newcommand\am{{\bf H}}
\newcommand\amd{\dot{\am}}
\newcommand\fo{{\bf F}}
\newcommand\po{\mathcal{P}}
\newcommand\xg{\ensuremath{\fv{R}}}
\newcommand\xgd{\ensuremath{\fvd{R}}}
\newcommand\xgdd{\ensuremath{\fvdd{R}}}
\newcommand\dvec[1]{\dot{\vec{ #1}}}
\newcommand\ddvec[1]{\ddot{\vec{ #1}}}
\newcommand\qp{\dot{q}}
\newcommand\dqp{\Delta \dot{q}}

\begin{document}


\section{La distorsion harmonique}
La distorsion est un critère de qualité en ce qui concerne les haut-parleurs.
Dans le soucis de construire un dispositif performant, nous avons décidé de 
nous informer sur la distorsion harmonique, un concept que nous ne connaissions
que de nom.
Cette section est structurée comme suit : nous parlerons tout d'abord de la notion  de distorsion 
en général pour ensuite aborder la notion  de distorsion 
harmonique, et finalement décrire ses causes, ses effets,
et les moyens de diminution.

\subsection{Définition}
Commençons tout d'abord par comprendre la notion de distorsion du son: par définition, c'est
une transformation du signal audio par rapport à celui de sortie. Une distorsion n'est généralement pas vraiment souhaitée, étant donné
que le signal en est déformé\cite{dico}. Cependant, certains audiophiles en tirent avantage, vu que que quelques
transformations peuvent mener à un son plus agréable\cite{encyclopedie}.

\paragraph{La distorsion harmonique}
La distorsion harmonique joue sur la superposition de différentes fréquences:
la fréquence fondamentale et ses harmoniques. Un haut-parleur parfait émettrait seulement la fréquence fondamentale, sans les harmoniques, qui sont donc des "parasites".
On parle d'harmoniques pour désigner les multiples entiers de la fréquence fondamentale\cite{encyclopedia}.
Par exemple, la seconde harmonique d'une fréquence de 50 Hz vaut 100Hz, la troisième 150Hz, etc. La figure 
ci-dessous illustre adéquatement cette notion.
Les harmoniques paires sont les moins incommodantes, étant donné qu'elles représentent la même note, mais à quelques octaves de différence.
Les harmoniques impaires, elles, sont plus gênantes étant donné que la note est différente\cite{hartmann}.


\begin{figure}[ht!]
\centering
\begin{tikzpicture}
\begin{axis}[
xtick=\empty,
ytick=\empty,
axis x line=middle,
axis y line=middle,
axis line style=->,
ylabel={$V$},
legend entries={ fréq. fondamentale, $3$ème harmonique, distorsion}
]
\addplot[samples=500,domain=0:2*pi, color=black]{sin(deg(x))};
\addplot[samples=500, domain=0:2*pi, color=orange]{-sin(3*deg(x))/3};
\addplot[samples=500, domain=0:2*pi, color=red]{sin(deg(x))-sin(3*deg(x))/3};
\end{axis}
\end{tikzpicture}
\caption{Superposition d'une fréquence fondamentale et de sa 3e harmonique}
\label{lwp_ratio} 
\end{figure}

\subsection{Causes}
À cause de la distorsion harmonique, le signal que nous faisons circuler dans notre haut-parleur n'est pas
une sinusoïdale parfaite mais plutôt une série de Fourier, c'est-à-dire une somme de sinusoïdes de 
fréquence et d'amplitude différentes. C'est l'appareil en lui-même qui crée la distorsion, à cause de la qualité
de certains composants\cite{cuccia} \cite{termans}.
Les charges non-linéaires sont les principales causes de distorsion harmonique. Celles-ci causent 
l'apparition des courants harmoniques qui sont eux-mêmes responsables de la distorsion harmonique. Elles sont 
surtout présentes dans la grande distribution d'électricité et ont posé problème autrefois, avant réglementation\cite{chargeslin}.

\subsection{Conséquences}
La distorsion harmonique a plusieurs conséquences néfastes.
La plus importante de toutes vient du fait que les harmoniques impaires génèrent un son dur, et peu agréable. De plus, la distorsion cause un accroissement 
du courant dans le système. Il va en résulter une surchauffe des composantes électriques (conducteurs, 
capacités,...). À la longue, des dysfonctionnements non souhaités peuvent provoquer un vieillissement 
précoce du circuit électrique\cite{brevet2}. Il existe de nombreuses autres conséquences néfastes, 
mais n'oublions pas de préciser que certaines personnes recherchent tout de même ces distorsions pour 
produire un son plus agréable, au moyen d'harmoniques paires. Dans le domaine de la musique, le timbre 
d'un instrument est déterminé par l'agencement des harmoniques qu'il produit. C'est ainsi qu'une même note
"sonne" différemment d'un instrument à un autre.

\subsection{Solutions}
Pour éviter toute distorsion, ou tout simplement pour émettre un son plus pur et exact, 
il existe différentes solutions. Nous parlerons seulement des filtres actifs 
même si de nombreuses autres pistes de solution ont été exploitées.
Les filtres actifs permettent d'éliminer les harmoniques perturbatrices en injectant des courants
harmoniques de mêmes fréquences mais déphasés d'une demi-période. Cela cause des interférences
destructrices avec les harmoniques dont on souhaite se débarrasser. La résultante est une droite constante
nulle n'influençant pas notre signal\cite{brevet1}.

% Just here to fix rapport_prejury.tex
\end{document}

\chapter{Conclusion}

Nous pouvons déjà jeter un regard en arrière sur le travail accompli au terme de ce pré-jury,
il reste pas mal de travail à faire évidemment mais nous ne sommes pas mécontents de nos 
avancements et nous sommes tous d'accord pour constater une nette amélioration par rapport au premier quadri. 
Nous avons pu améliorer notre rendement grâce à certains outils comme Dropbox, Latex ou Github 
et bien sûr l'expérience acquise lors du premier quadrimestre. Evidemment nous nous rendons compte 
que des progrès peuvent encore être réalisés.

\clearpage

\nocite{*} 
\bibliographystyle{plain}
\bibliography{../../sources/sources}

\part{Annexes}
\appendix

\documentclass{article}

\usepackage[utf8]{inputenc}
\usepackage[T1]{fontenc}      
\usepackage[francais]{babel}
\usepackage{graphicx}
\usepackage{circuitikz}
\usepackage[squaren, Gray]{SIunits}
\usepackage{sistyle}
\usepackage[autolanguage]{numprint}
\usepackage{pgfplots}
\usepackage{amsmath,amssymb,array}
\usepackage{url} 

% New command pour la modélisation mécanique, tri à effectuer
\newcommand\fv[1]{{\bf #1}} % free vector
\newcommand\fvd[1]{\dot{\bf #1}} % free vector derivated
\newcommand\fvdd[1]{\ddot{\bf #1}} % free vector derivated
\newcommand\fvr[1]{\mathring{\bf #1}} % free vector relatively derivated
\newcommand\fvrr[1]{\overset{\circ\circ}{\bf #1}} % free vector relatively derivated
\newcommand\uv[1]{{\bf\hat{ #1}}} % unit vector
\newcommand\ui{{\bf\hat{I}}} % unit vector I
\newcommand\uj{{\bf\hat{J}}} % unit vector J
\newcommand\uk{{\bf\hat{K}}} % unit vector K
\newcommand\wrt[2]{\ensuremath{\tensor*[_{ #1}]{ #2}{}}} % With Respect To
\newcommand\wtr[3]{\ensuremath{\tensor*[_{ #1}]{ #2}{^{ #3}}}} % With Two Respect
\newcommand\omegaf{{\bm \omega}}
\newcommand\omegafr{\mathring{\bm \omega}}
\newcommand\omegafd{\dot{\bm \omega}}
\newcommand\omegaft{\tilde{\bm \omega}}
\newcommand\omegaftr{\mathring{\tilde{\bm \omega}}}
\newcommand\omegat{\tilde{\omega}}
\newcommand\omegatd{\tilde{\dot{\omega}}}
\newcommand\ine{{\bf I}}
\newcommand\st{{\bf L}}
\newcommand\pst{{\bf M}}
\newcommand\lm{{\bf N}}
\newcommand\am{{\bf H}}
\newcommand\amd{\dot{\am}}
\newcommand\fo{{\bf F}}
\newcommand\po{\mathcal{P}}
\newcommand\xg{\ensuremath{\fv{R}}}
\newcommand\xgd{\ensuremath{\fvd{R}}}
\newcommand\xgdd{\ensuremath{\fvdd{R}}}
\newcommand\dvec[1]{\dot{\vec{ #1}}}
\newcommand\ddvec[1]{\ddot{\vec{ #1}}}
\newcommand\qp{\dot{q}}
\newcommand\dqp{\Delta \dot{q}}

\begin{document}


\section{Projet specifications}

\begin{table*} [h]

\begin{tabular}{|l|c|l|}

\hline
&&\\
\textbf{Group} & & \hfill \textbf{Date}  March 7th 2014\\
11.53 && \hfill \textbf{Version} 2.1\\

\hline
\multicolumn{3}{|p{15cm}|}{\textbf{Context} \newline
Our goal during this project is to realize, qualify, and measure an amplification system. This device should allow us to hear smartphone signals from two loudspeakers. The volume and the intensity of the bass and treble sounds should be adjustable.}  \\


\hline
\textbf{Date} & \textbf{Origine} & \textbf{Content}\\
\hline
&&\\
&&\textbf{Principal functions}\\
&&\\
16/02/14 & Costumer & 1. Emit a sound \\
16/02/14 & Costumer & 2. Amplify a sound \\
16/02/14 & Costumer & 3. Variation of bass and treble \\
&&\\
\hline
&&\\
& & \textbf{Criteria and level of the main functions} \\
&&\\
16/02/14 & Group & 1.1. Sound between ... and... Hz \\
16/02/14 & Costumer & 2.1. Power of 2.5 W \\
&&\\
\hline
&&\\
& & \textbf{Constraints} \\
&&\\
16/02/14 & Costumer &   Jackplug of 3.5 mm in diameter\\
16/02/14 & Laboratory &  Input voltage of 30V \\
07/03/14 & Costumer&  Paper membrane \\
&&\\
\hline
&&\\
& & \textbf{Terms} \\
&&\\
07/03/14 & Group & Type of paper : 200 g/m$^{2}$ \\
07/03/14 & Group & Cost estimation : ...\\

&&\\
\hline
\end{tabular}

\end{table*}

\end{document}


\bigbreak

\documentclass{article}

\usepackage[utf8]{inputenc}
\usepackage[T1]{fontenc}      
\usepackage[francais]{babel}
\usepackage{graphicx}
\usepackage{circuitikz}
\usepackage[squaren, Gray]{SIunits}
\usepackage{sistyle}
\usepackage[autolanguage]{numprint}
\usepackage{pgfplots}
\usepackage{amsmath,amssymb,array}
\usepackage{url} 

% New command pour la modélisation mécanique, tri à effectuer
\newcommand\fv[1]{{\bf #1}} % free vector
\newcommand\fvd[1]{\dot{\bf #1}} % free vector derivated
\newcommand\fvdd[1]{\ddot{\bf #1}} % free vector derivated
\newcommand\fvr[1]{\mathring{\bf #1}} % free vector relatively derivated
\newcommand\fvrr[1]{\overset{\circ\circ}{\bf #1}} % free vector relatively derivated
\newcommand\uv[1]{{\bf\hat{ #1}}} % unit vector
\newcommand\ui{{\bf\hat{I}}} % unit vector I
\newcommand\uj{{\bf\hat{J}}} % unit vector J
\newcommand\uk{{\bf\hat{K}}} % unit vector K
\newcommand\wrt[2]{\ensuremath{\tensor*[_{ #1}]{ #2}{}}} % With Respect To
\newcommand\wtr[3]{\ensuremath{\tensor*[_{ #1}]{ #2}{^{ #3}}}} % With Two Respect
\newcommand\omegaf{{\bm \omega}}
\newcommand\omegafr{\mathring{\bm \omega}}
\newcommand\omegafd{\dot{\bm \omega}}
\newcommand\omegaft{\tilde{\bm \omega}}
\newcommand\omegaftr{\mathring{\tilde{\bm \omega}}}
\newcommand\omegat{\tilde{\omega}}
\newcommand\omegatd{\tilde{\dot{\omega}}}
\newcommand\ine{{\bf I}}
\newcommand\st{{\bf L}}
\newcommand\pst{{\bf M}}
\newcommand\lm{{\bf N}}
\newcommand\am{{\bf H}}
\newcommand\amd{\dot{\am}}
\newcommand\fo{{\bf F}}
\newcommand\po{\mathcal{P}}
\newcommand\xg{\ensuremath{\fv{R}}}
\newcommand\xgd{\ensuremath{\fvd{R}}}
\newcommand\xgdd{\ensuremath{\fvdd{R}}}
\newcommand\dvec[1]{\dot{\vec{ #1}}}
\newcommand\ddvec[1]{\ddot{\vec{ #1}}}
\newcommand\qp{\dot{q}}
\newcommand\dqp{\Delta \dot{q}}

\begin{document}


\section{Méthode de recherche}

\subsection{La contre-réaction ou réaction négative}
Comme suggeré lors de la séance d'information sur la recherche bibliographique,
nous avons appliqué la méthode de l'entonnoir. Comme les boucles de contre-réaction 
sont directement liées aux amplificateurs, nous avons commencé nos recherches avec 
les termes plutôt généraux : \textit{amplificateurs} et \textit{amplifiers}. Nous 
nous avons ensuite associé à ces mots clés les termes plus précis : \textit{contre-réaction}
et \textit{negative feedback}.

Les différents ouvrages et documents que nous avons utilisés sont listés dans la bibliographie.

\subsection{La distorsion harmonique}

\paragraph{Choix du thème}
Le choix du thème n'a pas été chose aisée. Nous avons commencé par établir un brainstorming afin de réunir 
le plus d'idées possibles. Cependant, les thèmes proposés nous semblaient trop généraux que pour faire un vrai 
travail en profondeur tout en restant concis. Quelqu'un a finalement proposé la distorsion harmonique ; un 
terme visible sur les emballages de haut-parleurs. Nous avions également repéré ce terme dans la datasheet 
de l'amplificateur audio reçu pour le projet: une valeur de \numprint{0.2\%} était renseignée pour le THD 
(taux de distorsion harmonique). Curieux d'en apprendre plus sur ce terme presque méconnu, 
nous avons décidé de débuter notre travail de recherche là-dessus.

\paragraph{Recherche documentaire}
Etant donné que nous ne connaissions vraiment que très peu sur ce sujet et que nous 
devions le comprendre en profondeur, nous avons commencé par le terme général de "distorsion".
Une première recherche sur internet a permis de fixer les idées à propos de ce terme, et nous 
avons ensuite pu établir une liste de mot-clefs pour entamer réellement la recherche sur la 
distorsion harmonique. Nous avons appliqué la "technique de l'entonnoir", et nous avons finalement 
réuni assez d'informations que pour écrire ce rapport. Notons tout de même que c'est indiscutablement
en anglais que nous avons 
trouvé le plus d'informations. Nous avons gardé une trace de toutes les sources que nous avons 
consultées, et cela a rendu l'écriture de la bibliographie nettement plus facile.

% Just here to fix rapport_prejury.tex
\end{document}


\end{document}
