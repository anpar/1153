\input{../head.tex}

\section{Dimensionnement du haut-parleur}
Après avoir réalisé quelques recherches sur les haut-parleurs, nous avons pu imaginer le dispositif idéal 
à réaliser. En tenant compte des différentes contraintes qui nous étaient imposées, voici les différents 
choix que nous avons effectués.

\subsection{Pour le boîtier}
Nous devions pouvoir faire varier les fréquences (voir annexe "Cahier des Charges"), ce qui signifie que nous ne 
pouvions pas faire un caisson trop petit. La taille du caisson influence le son restitué par le haut-parleur :
un volume trop petit ne restituerait pas les extrêmes graves. En effet, à de très basses 
fréquences, l'air à  l'interieur du haut-parleur chauffe et la pression augmente.  
L'enceinte close va se comporter comme une raideur supplémentaire qui augmente la fréquence de
résonance\footnote{Fréquence pour laquelle la réponse du circuit est maximale}.  Comme la fréquence de raisonnance se 
situe au milieu de la fréquence de coupure du passe-haut et du passe-bas, si nous augmentons la fréquence de résonnance, 
on augmente la fréquence de coupure du passe-haut\cite{volume} également. Nous avons finalement opté pour un boîtier cubique de 
$\unit{25}{\centi\meter}$ de côté, c'est souvent la forme d'un haut-pareur qu'on peut retrouver dans le commerce.
Nous avions pensé placer un évent à l'avant du haut-parleur pour augmenter le rendement en profitant de l'onde 
arrière, mais c'était plus difficile à construire, et il aurait fallu que l'on accorde l'event, de manière à
exploiter l'onde arrière correctement. Nous nous sommes donc finalement limités à une charge
\footnote{Manière de séparer les ondes avant et arrière.} dite "\textit{close}"\cite{close}.  

Afin d'améliorer un peu le boîtier, nous avons également pensé aux éléments suivants :

\begin{itemize}
	\item	Des pieds en caoutchouc : placer des pieds en caoutchouc sur le boîtier de notre haut-parleur
				permet de réduire les déplacements dûs aux vibrations du haut-parleur ;
	\item	Un bois épais pour le caisson pour empecher le haut-parleur de bouger avec les ondes et éviter que 
	le son produit ne sorte par autre chose que la membrane.
	
\end{itemize}

\subsection{Pour la membrane}
Nous avons opté pour une membrane de diamètre de $\unit{17}{\centi\meter}$. Nous avons choisi cette valeur afin 
d'avoir une membrane assez large, pour exploiter le mieux possible la taille du caisson. C'est également un
diamètre assez répandu dans le commerce\cite{tlhp}. Nous respectons donc les normes.
La profondeur de la membrane est de $\unit{6}{\centi\meter}$, comme pour la plupart des membranes de ce
diamètre\cite{tlhp}. La membrane est réalisée en papier. Pour avoir une constante de raideur nous avons eu besoin
d'un ressort, nous l'avons réalisé en tissus tendu, ce qui est assez inovant pour un haut-parleur.

\paragraph{Tableau récapitulatif}

\begin{table}
	\centering
	\begin{tabularx}{\textwidth}{|X|X|}
		\hline
			 \textbf{Caractéristique} & \textbf{Justification} \\
		\hline
			Volume du caisson : $\unit{25\times25\times25}{\centi\meter}$ & Possibilité de faire varier les fréquences.  \\
		\hline
			Matériau du caisson : Panneau de 	MDF
			d'épaisseur \unit{18}{\milli\meter} & Qualité, robustesse et coût. \\
		\hline
			Diamètre de membrane : \unit{17}{\centi\meter} & Avoir une membrane assez large pour exploiter le mieux possible la taille du caisson. \\
		\hline
			Profondeur de la membrane : \unit{6}{\centi\meter} & Déterminé en fonction du diamètre de la membrane. \\
		\hline
			Materiau membrane : papier et tissus & Rigidité et petite constante de raideur. \\
		\hline
			Masse surfacique du papier : \unit{200}{\gram\per\meter\squared} & Rigidité et coût. \\
		\hline
	\end{tabularx}
\end{table}

\input{../foot.tex}
