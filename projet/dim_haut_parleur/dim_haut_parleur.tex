\input{../head.tex}

\section{Dimensionnement du haut-parleur}
Après avoir réalisé quelques recherches sur les haut-parleurs, nous avons pu imaginer le dispositif idéal 
à réaliser. En tenant compte des différentes contraintes qui nous étaient imposées, voici les différents 
choix que nous avons effectués.

\subsection{Pour le boîtier}
La première question qui s'est posée était celle du volume du caisson. Or le volume du caisson ($V_b$) est 
lié à la fréquence de résonance\footnote{Fréquence pour laquelle la réponse du circuit est maximale} du 
haut-parleur à l'air libre ($F_s$) , la fréquence de résonance du haut-parleur fermé ($F_b$), et au volume 
d'air équivalent à la suspension du haut-parleur\footnote{Unité pour définir la compliance de la suspension 
en fonction de la surface de la membrane. Exprimé en litres, une grande valeur indique une suspension 
souple\cite{?}.} ($V_{as}$) , selon l'équation suivante\cite{?}:

$$\frac{F_b}{F_s} = \sqrt(\frac{V_{as}}{V_b} +1)$$

Nous remarquons qu'un haut-parleur idéal serait de volume infini, étant donné que sa fréquence de résonance 
se rapprocherait de celle à l'air libre. Mais nous ne voulions pas d'un caisson trop grand, pour des questions pratiques et esthétiques.
Le fait que la fréquence de résonance d'un haut-parleur est plus élevée que celle à l'air libre implique que
la fréquence de coupure du passe-haut est également plus élevée. Cela a pour conséquence
qu'un volume de caisson trop petit ne restitue pas les extrêmes graves. Nous avons finalement opté pour un boîtier 
cubique de $\unit{25}{\centi\meter}$ de côté.   

Afin d'améliorer un peu le boîtier, nous avons également pensé à placer des pieds en caoutchouc afin de
réduire les déplacements dûs aux vibrations du haut-parleur. Nous avions également pensé placer un évent à l'avant du haut-parleur 
pour augmenter le rendement en profitant de l'onde arrière, mais c'était plus difficile à construire, et 
il aurait fallu que l'on accorde l'event, de manière à exploiter l'onde arrière correctement. Nous nous 
sommes donc finalement limités à une charge\footnote{Manière de séparer les ondes avant et arrière.} dite 
"\textit{close}"\cite{close}.


\subsection{Pour la membrane}
Nous avons opté pour une membrane de diamètre de $\unit{17}{\centi\meter}$. Nous avons choisi cette valeur afin 
d'avoir une membrane assez large, pour exploiter le mieux possible la taille du caisson. C'est également un
diamètre assez répandu dans le commerce\cite{tlhp}. Nous respectons donc les normes.
La profondeur de la membrane est de $\unit{6}{\centi\meter}$, comme pour la plupart des membranes de ce
diamètre\cite{tlhp}. Elle est réalisée en papier, et nous avons opté pour du tissus tendu en guise de ressort.
Cette solution nous apparaît comme sortant de l'ordinaire, propre, et efficace. Cela nous a en effet permis
d'obtenir une constante de raideur minime.


\paragraph{Tableau récapitulatif}

\begin{table}
	\centering
	\begin{tabularx}{\textwidth}{|X|X|}
		\hline
			 \textbf{Caractéristique} & \textbf{Justification} \\
		\hline
			Volume du caisson : $\unit{25\times25\times25}{\centi\meter}$ & Possibilité de faire varier les fréquences.  \\
		\hline
			Matériau du caisson : Panneau de 	MDF
			d'épaisseur \unit{18}{\milli\meter} & Qualité, robustesse et coût. \\
		\hline
			Diamètre de membrane : \unit{17}{\centi\meter} & Avoir une membrane assez large pour exploiter le mieux possible la taille du caisson. \\
		\hline
			Profondeur de la membrane : \unit{6}{\centi\meter} & Déterminé en fonction du diamètre de la membrane. \\
		\hline
			Materiau membrane : papier et tissus & Rigidité et petite constante de raideur. \\
		\hline
			Masse surfacique du papier : \unit{200}{\gram\per\meter\squared} & Rigidité et coût. \\
		\hline
	\end{tabularx}
\end{table}

\input{../foot.tex}
