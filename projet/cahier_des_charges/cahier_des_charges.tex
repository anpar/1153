\documentclass{article}

\usepackage[utf8]{inputenc}
\usepackage[T1]{fontenc}      
\usepackage[francais]{babel}
\usepackage{graphicx}
\usepackage{circuitikz}
\usepackage[squaren, Gray]{SIunits}
\usepackage{sistyle}
\usepackage[autolanguage]{numprint}
\usepackage{pgfplots}
\usepackage{amsmath,amssymb,array}
\usepackage{url} 

% New command pour la modélisation mécanique, tri à effectuer
\newcommand\fv[1]{{\bf #1}} % free vector
\newcommand\fvd[1]{\dot{\bf #1}} % free vector derivated
\newcommand\fvdd[1]{\ddot{\bf #1}} % free vector derivated
\newcommand\fvr[1]{\mathring{\bf #1}} % free vector relatively derivated
\newcommand\fvrr[1]{\overset{\circ\circ}{\bf #1}} % free vector relatively derivated
\newcommand\uv[1]{{\bf\hat{ #1}}} % unit vector
\newcommand\ui{{\bf\hat{I}}} % unit vector I
\newcommand\uj{{\bf\hat{J}}} % unit vector J
\newcommand\uk{{\bf\hat{K}}} % unit vector K
\newcommand\wrt[2]{\ensuremath{\tensor*[_{ #1}]{ #2}{}}} % With Respect To
\newcommand\wtr[3]{\ensuremath{\tensor*[_{ #1}]{ #2}{^{ #3}}}} % With Two Respect
\newcommand\omegaf{{\bm \omega}}
\newcommand\omegafr{\mathring{\bm \omega}}
\newcommand\omegafd{\dot{\bm \omega}}
\newcommand\omegaft{\tilde{\bm \omega}}
\newcommand\omegaftr{\mathring{\tilde{\bm \omega}}}
\newcommand\omegat{\tilde{\omega}}
\newcommand\omegatd{\tilde{\dot{\omega}}}
\newcommand\ine{{\bf I}}
\newcommand\st{{\bf L}}
\newcommand\pst{{\bf M}}
\newcommand\lm{{\bf N}}
\newcommand\am{{\bf H}}
\newcommand\amd{\dot{\am}}
\newcommand\fo{{\bf F}}
\newcommand\po{\mathcal{P}}
\newcommand\xg{\ensuremath{\fv{R}}}
\newcommand\xgd{\ensuremath{\fvd{R}}}
\newcommand\xgdd{\ensuremath{\fvdd{R}}}
\newcommand\dvec[1]{\dot{\vec{ #1}}}
\newcommand\ddvec[1]{\ddot{\vec{ #1}}}
\newcommand\qp{\dot{q}}
\newcommand\dqp{\Delta \dot{q}}

\begin{document}


\section{Cahier des charges}

\begin{table*} [h]

\begin{tabular}{|l|c|l|}

\hline
&&\\
\textbf{Groupe} & & \hfill \textbf{Date} 7 mars 2014\\
11.53 && \hfill \textbf{Version} 2.1\\

\hline
\multicolumn{3}{|p{15cm}|}{\textbf{Contexte} \newline
Notre objectif, dans le cadre de notre projet, est de réaliser, mesurer et qualifier un dispositif comportant un système d'amplification permettant  d'écouter sur deux haut-parleurs les signaux stéré provenant d'un smartphone et d'en faire varier le volume, l'intensité des sons graves et aigus.}  \\


\hline
\textbf{Date} & \textbf{Origine} & \textbf{Contenu}\\
\hline
&&\\
&&\textbf{Fonctions principales}\\
&&\\
16/02/14 & Client & 1. Emettre un son \\
16/02/14 & Client & 2. Amplifier un son \\
16/02/14 & Client & 3. Variation des sons graves et aigus \\
&&\\
\hline
&&\\
& & \textbf{Critères et niveaux des fonctions principales} \\
&&\\
16/02/14 & Groupe & 1.1. Son entre ... et ... Hz \\
16/02/14 & Client & 2.1. Puissance de 2,5 W \\
&&\\
\hline
&&\\
& & \textbf{Containtes} \\
&&\\
16/02/14 & Client &  Entrée du son avec une prise Jack de 3.5 mm de diamètre \\
16/02/14 & Labo &  Tension d'entrée de 30V \\
07/03/14 & Client &  Cône du baffle en matériau papier \\
&&\\
\hline
&&\\
& & \textbf{Modalités} \\
&&\\
07/03/14 & Groupe & Type de papier : ... g/m$^{2}$ \\
07/03/14 & Groupe & Estimation du prix : ...\\

&&\\
\hline
\end{tabular}

\end{table*}

\end{document}
