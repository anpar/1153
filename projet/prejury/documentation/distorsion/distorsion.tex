% Ce fichier sera inclut automatiquement dans le rapport final via une commande LaTex
% Il est donc inutile d'indiquer l'ent�te habituelle, il suffit de commencer par \section{}
\documentclass{article}

\usepackage[latin1]{inputenc} 
\usepackage[T1]{fontenc}      
\usepackage[francais]{babel} 


\begin{document}

\section{Recherche documentaire: la distorsion harmonique}


La distortion est un critère de qualité en ce qui concerne les haut-parleurs. Dans le soucis de construire un dispositif de qualité, nous avons décidé de nous informer sur la distorsion harmonique, un concept que nous ne connaissions que de nom.
Ce document est structuré comme suit: Nous parlerons tout d'abord de la méthode de recherche que nous avons adoptée, pour ensuite aborder la notion  de distorsion en général, et finalement décrire la distorsion harmonique, ses causes, ses effets, et les moyens de diminution.

\paragraph{Méthode de recherche}
Etant donné que nous ne connaissions vraiment que très peu sur ce sujet et que nous le devions comprendre en profondeur, nous avons commencé par le terme général de "distorsion". Une première recherche sur internet a permis de fixer les idées à propos de ce thème, et nous avons ensuite pu établir une liste de mot-clefs pour entamer réellement la recherche sur la distorsion harmonique. Nous avons appliqué la "technique de l'entonnoir", et nous avons finalement réuni assez d'informations que pour écrire ce rapport. L'encyclopédie \textit{Universalis} nous a été d'une grande aide. Notons tout de même que c'est indiscutablement en anglais que nous avons trouvé le plus d'informations. Nous avons gardé une trace de toutes les sources que nous avons consultées, et cela a rendu l'écriture de la bibliographie nettement plus facile.



\subsection{Définition}
Commençons tout d'abord par comprendre la notion de distorsion du son: par définition, c'est est une transformation d'un signal audio. Ces distorsions ne sont pas souvent recherchées, étant donné que le signal en est déformé. Cependant, certains audiophiles en tirent avantage, vu que que quelques transformations peuvent mener à un son plus chaud et agréable.

\paragraph{La distorsion harnomique}La distorsion harmonique est, par exemple, une distorsion responsable d'un son plus "chaud", et agréable à écouter. Elle joue sur l'ajout de différentes fréquences, se supperposant à la fréquence fondamentale. On parle de deuxième, troisième,... harmonique lorque la fréquence sortante vaut deux, trois,... fois la fréquence entrante. Elles s'organisent en deux familles: Les harmoniques paires qui correspondent aux octaves (H2 pour la 1ère octave, H4 pour la 2ème octave, etc...) et qui tendent à "arrondir" les sonorités et leur donnant un coté "chaleureux".
- Les harmoniques impaires qui elles tendent plutôt à engendrer un son "dissonant" et sont souvent à l'origine de "duretés". Elles ont ainsi tendance a accentuer les défauts intrinsèques des haut-parleurs.
A REFORMULER + SCHEMA?


\subsection{Causes}
Une des causes principales de la distorsion harmonique est le haut parleur, qui ajoute des distorsions au signal. Les vibrations dans l'enceinte sont également uen source de distorsion du signal. A COMPLETER



\subsection{Conséquences}
Lorsque les fréquences sont correctement ajustées, l'ajout d'harmoniques peut rendre le son plus chaud et agréable à écouter. C'est pour cela que certains la recherchent.
Ces distorsions sont parfois peu audibles, et un certain pourcentage est tout à fait acceptable puisuqe notre oreille y est tolérante.
A COMPLETER

\subsection{Solutions}
Pour éviter les mauvaises distorsion, ou tout simplement pour émettre un son pur et exact, il existe différentes solutions.
A COMPLETER


%Si vous injectez un signal sinusoïdal dans un haut-parleur, l'onde acoustique émise par le haut-parleur contiendra la fréquence fondamentale et différentes harmoniques. Ces dernières correspondent à la distorsion. Le rapport de puissance entre les harmoniques et la fréquence fondamentale est le taux de distorsion. 
%http://www.audiophile-scientifique.com/theorie-acoustique/distorsion/distorsion-harmonique.php

\bibliographystyle{plain}
\bibliography{source}


\end{document}
