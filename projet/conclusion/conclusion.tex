\input{../head.tex}

% A retravailler d'après mon relecteur

En conclusion, nous avons compris le fonctionnement d’un haut-parleur, 
nous avons appris à travailler en groupe, de façon organisée et responsable
et nous avons essayé de fabriquer notre propre haut-parleur. Nous pouvions clairement observer, sur l'oscilloscope,
la musique à la sortie de notre circuit imprimé. Malheureusement, en toute objectivité, aucun son n'a été reproduit par 
notre haut-parleur.

Bien que notre haut-parleur ne respecte pas tous les objectifs fixés dans le cahier des charges, nous pouvons
néanmoins souligner de nombreux points positifs concernant notre projet.
Nous avons compris et assimilé le fonctionnement théorique d’un haut-parleur, 
nous avons modélisé un haut-parleur avec des règles physiques et mathématiques 
telles que les lois du magnétisme, l'approximation mathématique,etc. Nous avons également
caractérisé la puissance, la physionomie ainsi que toutes les composantes internes dudit haut-parleur. 
Concernant le travail de groupe, nous pouvons affirmer que le groupe est resté soudé, tout comme les PCB,
pendant tout le quadrimestre.  Tout le monde était là à chaque séance sauf en cas
de force majeure.  Tout le monde a apporté sa contribution au projet, nous avons essayé de
partager les tâches le mieux possible même s’il n’est pas possible d’avoir 
exactement même charge de travail pour chacun.  Le point faible de notre groupe est 
que nous ne privilégions pas assez les réunions réelles, nous travaillons de notre 
côté et puis seulement nous mettons en commun et parfois le même travail était 
réalisé plusieurs fois.  Un meilleur rendement aurait fait avancer le projet plus 
rapidement et plus intelligemment. 
Le fait que nous utilisions les mêmes outils ont facilité la mise en commun et l’
échange de documents et d’information. Par exemple, nous avons fait l’effort d’apprendre 
\LaTeX durant ce quadrimestre pour écrire notre rapport; nous avons aussi
créé une dropbox ainsi qu’on compte GitHub où tous les documents étaient modifiables à 
souhait. Pour la plupart des membres du groupe, le projet était 
plus structuré dans notre groupe actuel que dans nos anciens groupes. Nous nous voyions aussi 
assez régulièrement pour mettre nos idées en commun: malgré tous les outils à notre 
disposition, la réunion physique reste quand même le meilleur moyen de communiquer.
Grâce au planning réalisé lors du pré-jury, nous avons su avancer dans le projet de manière
organisée et claire.  Malgré quelques écarts, nous nous sommes assez bien tenus au plan.
Il nous a bien servi pour acquérir une vision claire et structurée du projet. 

Pour terminer, les concepts mathématiques et physiques ont bien été assimilés mais en 
pratique, le haut-parleur ne fonctionnait pas comme nous le souhaitions.  Il y a eu des
problèmes entre la théorie et la pratique malgré les tests que nous avons faits pour que 
le haut-parleur fonctionne aussi bien qu’il devrait théoriquement le faire.  Le groupe 
est néanmoins resté solidaire durant tout le quadrimestre. Ce projet nous aura donc été
profitable, et c'est avec une grande fierté que nous y apportons le point final.

\input{../foot.tex}
