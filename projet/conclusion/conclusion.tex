\documentclass{article}

\usepackage[utf8]{inputenc}
\usepackage[T1]{fontenc}      
\usepackage[francais]{babel}
\usepackage{graphicx}
\usepackage{circuitikz}
\usepackage[squaren, Gray]{SIunits}
\usepackage{sistyle}
\usepackage[autolanguage]{numprint}
\usepackage{pgfplots}
\usepackage{amsmath,amssymb,array}
\usepackage{url} 

% New command pour la modélisation mécanique, tri à effectuer
\newcommand\fv[1]{{\bf #1}} % free vector
\newcommand\fvd[1]{\dot{\bf #1}} % free vector derivated
\newcommand\fvdd[1]{\ddot{\bf #1}} % free vector derivated
\newcommand\fvr[1]{\mathring{\bf #1}} % free vector relatively derivated
\newcommand\fvrr[1]{\overset{\circ\circ}{\bf #1}} % free vector relatively derivated
\newcommand\uv[1]{{\bf\hat{ #1}}} % unit vector
\newcommand\ui{{\bf\hat{I}}} % unit vector I
\newcommand\uj{{\bf\hat{J}}} % unit vector J
\newcommand\uk{{\bf\hat{K}}} % unit vector K
\newcommand\wrt[2]{\ensuremath{\tensor*[_{ #1}]{ #2}{}}} % With Respect To
\newcommand\wtr[3]{\ensuremath{\tensor*[_{ #1}]{ #2}{^{ #3}}}} % With Two Respect
\newcommand\omegaf{{\bm \omega}}
\newcommand\omegafr{\mathring{\bm \omega}}
\newcommand\omegafd{\dot{\bm \omega}}
\newcommand\omegaft{\tilde{\bm \omega}}
\newcommand\omegaftr{\mathring{\tilde{\bm \omega}}}
\newcommand\omegat{\tilde{\omega}}
\newcommand\omegatd{\tilde{\dot{\omega}}}
\newcommand\ine{{\bf I}}
\newcommand\st{{\bf L}}
\newcommand\pst{{\bf M}}
\newcommand\lm{{\bf N}}
\newcommand\am{{\bf H}}
\newcommand\amd{\dot{\am}}
\newcommand\fo{{\bf F}}
\newcommand\po{\mathcal{P}}
\newcommand\xg{\ensuremath{\fv{R}}}
\newcommand\xgd{\ensuremath{\fvd{R}}}
\newcommand\xgdd{\ensuremath{\fvdd{R}}}
\newcommand\dvec[1]{\dot{\vec{ #1}}}
\newcommand\ddvec[1]{\ddot{\vec{ #1}}}
\newcommand\qp{\dot{q}}
\newcommand\dqp{\Delta \dot{q}}

\begin{document}


+En conclusion, nous avons compris le fonctionnement d’un haut-parleur, 
nous avons appris à travailler en groupe, de façon organisée et responsable
et nous avons essayé de fabriquer notre propre haut-parleur. 
En toute objectivité, aucune musique ne sortait de notre haut-parleur.  
Cependant nous voyions clairement grâce à l’oscilloscope la fréquence de 
la musique à la sortie du circuit imprimé.
Si l’on peut dire que la fabrication du haut-parleur est un échec, on peut
néanmoins voir de nombreux points positifs concernant notre projet.
Nous avons compris et assimilé le fonctionnement théorique d’un haut-parleur, 
nous avons modélisé un haut-parleur avec des règles physiques et mathématiques 
tel que les lois du magnétisme, l’approximation mathématique,…  Nous avons également
su caractérisé la puissance, la taille, toutes les composantes internes d’un haut-parleur. 
Concernant le travail de groupe, nous pouvons affirmer que le groupe est resté soudé
pendant tout le quadrimestre.  Tout le monde était là à chaque séance sauf en cas
de force majeure.  Tout le monde a travaillé sur le projet, nous avons essayé de
partager les tâches le mieux possible même s’il n’est pas possible d’avoir 
exactement tous la même charge de travail.  Le point faible de notre groupe est 
que nous ne privilégions pas assez les réunions réels, nous travaillons de notre 
côté et puis seulement nous mettons en commun et parfois le même travail était 
réalisé plusieurs fois.  Un meilleur rendement aurait fait avancer le projet plus 
rapidement et plus intelligemment. 
Le fait que nous utilisions les mêmes outils ont facilité la mise en commun et l’
échange de document et d’information. Par exemple, nous avons fait l’effort d’apprendre 
\latex{} durant ce quadrimestre pour pouvoir écrire notre rapport avec, nous avons aussi
créé une dropbox ainsi qu’on compte GitHub où tous les documents étaient modifiables à 
tout moment du jour et de la nuit.  Pour la plupart des membres du groupe, le projet était 
plus structuré dans notre groupe que dans nos anciens groupes.  Nous nous voyions aussi 
assez régulièrement pour mettre nos idées en communs, malgré tous les outils à notre 
disposition, la réunion physique reste quand même le meilleur moyen de communiquer.
Pour terminer, les concepts mathématiques et physiques ont bien été assimilés mais en 
pratique, le haut-parleur ne fonctionnait pas comme nous le souhaitions.  Il y a eu des
problèmes entre la théorie et la pratique malgré les tests que nous avons faits pour que 
le haut-parleur fonctionne aussi bien qu’il devrait théoriquement le faire.  Le groupe 
est néanmoins resté solidaire durant tout le quadrimestre.

% Just here to fix rapport_prejury.tex
\end{document}
