\documentclass{article}

\usepackage[utf8]{inputenc}
\usepackage[T1]{fontenc}      
\usepackage[francais]{babel}
\usepackage{graphicx}
\usepackage{circuitikz}
\usepackage[squaren, Gray]{SIunits}
\usepackage{sistyle}
\usepackage[autolanguage]{numprint}
\usepackage{pgfplots}
\usepackage{amsmath,amssymb,array}
\usepackage{url} 

% New command pour la modélisation mécanique, tri à effectuer
\newcommand\fv[1]{{\bf #1}} % free vector
\newcommand\fvd[1]{\dot{\bf #1}} % free vector derivated
\newcommand\fvdd[1]{\ddot{\bf #1}} % free vector derivated
\newcommand\fvr[1]{\mathring{\bf #1}} % free vector relatively derivated
\newcommand\fvrr[1]{\overset{\circ\circ}{\bf #1}} % free vector relatively derivated
\newcommand\uv[1]{{\bf\hat{ #1}}} % unit vector
\newcommand\ui{{\bf\hat{I}}} % unit vector I
\newcommand\uj{{\bf\hat{J}}} % unit vector J
\newcommand\uk{{\bf\hat{K}}} % unit vector K
\newcommand\wrt[2]{\ensuremath{\tensor*[_{ #1}]{ #2}{}}} % With Respect To
\newcommand\wtr[3]{\ensuremath{\tensor*[_{ #1}]{ #2}{^{ #3}}}} % With Two Respect
\newcommand\omegaf{{\bm \omega}}
\newcommand\omegafr{\mathring{\bm \omega}}
\newcommand\omegafd{\dot{\bm \omega}}
\newcommand\omegaft{\tilde{\bm \omega}}
\newcommand\omegaftr{\mathring{\tilde{\bm \omega}}}
\newcommand\omegat{\tilde{\omega}}
\newcommand\omegatd{\tilde{\dot{\omega}}}
\newcommand\ine{{\bf I}}
\newcommand\st{{\bf L}}
\newcommand\pst{{\bf M}}
\newcommand\lm{{\bf N}}
\newcommand\am{{\bf H}}
\newcommand\amd{\dot{\am}}
\newcommand\fo{{\bf F}}
\newcommand\po{\mathcal{P}}
\newcommand\xg{\ensuremath{\fv{R}}}
\newcommand\xgd{\ensuremath{\fvd{R}}}
\newcommand\xgdd{\ensuremath{\fvdd{R}}}
\newcommand\dvec[1]{\dot{\vec{ #1}}}
\newcommand\ddvec[1]{\ddot{\vec{ #1}}}
\newcommand\qp{\dot{q}}
\newcommand\dqp{\Delta \dot{q}}

\begin{document}


\section{Dimensionnement de l'électroaimant et de la bobine mobile}
Pour fabriquer notre haut-parleur, nous ne disposions pas d'aimant permanent. Nous avons donc
dû créer un électroaimant à partir d'un matériau ferromagnétique qui nous a été fourni.
Cette section présente dans un premier temps le dimensionnement de cet électroaimant, c'est-à-dire le
nombre de spires choisi, la résistance totale de la bobine, etc.

Nous calculerons ensuite, de manière expérimentale, la constante de raideur de la membrane de
notre haut-parleur. A partir de cela et de l'écartement maximal par rapport à sa position d'origine 
(choisi arbitrairement), 
nous pourrons calculer la force nécessaire pour déplacer la membrane, et par conséquent, le nombre
de spires nécessaire sur la bobine mobile.

\subsection{Fonctionnement et dimensionnement de l'électroaimant}
Lorsqu'un courant traverse la bobine de cuivre, un champ magnétique est créé.  Nous obtenons 
donc un électroaimant fixe générant le champ nécessaire au déplacement de la seconde bobine. 
C'est cette seconde bobine qui sera responsable du tremblement de la membrane.

\begin{figure}[ht!]
\centering
\includegraphics[scale=0.6]{electroaimant.png}
\caption{Modélisation d'un électroaimant}
\label{modélisation de l'électroaimant}
\end{figure}

Dans le cas de notre haut-parleur, la bobine créant le champ magnétique est situé sur 
la branche centrale d'un matériau magnétique en forme de ''E''. La forme du circuit 
magnétique permet de concenter l'effet du champ magnétique dans l'entrefer. Le matériau
magnétique est constitué de fines lamelles empilées les unes sur les autres afin
d'éviter les pertes dûes au courant de \textsc{Foucault}.

Le nombre de spires de la bobine fixe, appelons-le $N_1$, a été choisi arbitrairement de manière à produire un
champ magnétique assez fort. Nous avons fixé ce nombre, selon les conseils de notre tuteur, à \numprint{420}. 
Nous allons maintenant calculer les caractéristiques suivantes de notre électroaimant :

\begin{itemize}
	\item Nombre de spires;
	\item Résistance totale de la bobine ;
	\item Champ magnétique induit ;
\end{itemize}

\paragraph{Champ magnétique dans l'entrefer}
Pour céer un champ magnétique plus fort, nous avons réduit l'entrefer de $\unit{4}{\milli\meter}$.
Calculons dans un premier temps le champ magnétique dans l'entrefer en 
utilisant la conservation des flux. Pour ce calcul, nous utilisons l'hypothèse simplificatrice
assez forte que tout le champ se trouve dans l'entrefer.

$$H_e \cdot e = N_1 I \Rightarrow \frac{B_e}{\mu_0 \mu_r} e = N_1 I$$

Pour $N_1 = 420$, l'entrefer $e = \unit{0.007}{\meter}$, $\mu_r = \unit{1.0000004}{\frac{\henry}{\meter}}$ la perméabilité magnétique
de l'air et $I = \unit{1}{\ampere}$, nous trouvons alors :

$$B_e = \unit{0.07539825}{\tesla}$$

\paragraph{Résistance totale de la bobine}
Pour calculer la résistance totale de la bobine, nous devons connaître la longueur totale de fil de cuivre utilisé.
Pour cela nous utilisons la formule suivante :

$$L_{fil} = N_1 \cdot 2\pi r$$  

Dans cette formule, on fait l'hypothèse que les spires qui constituent la bobine forment des cercles
parfaits.
Où $N_1 = 420$ est le nombre de spires de la bobine fixe, et $r$ est la rayon des spires. Pour
$r = \unit{0.016}{\meter}$, nous trouvons :

$$L_{fil} = \unit{42.22}{\meter}$$

Il ne nous reste donc plus qu'à multiplier la longueur totale trouvée par la résistance linéique des fils de cuivre
($R_{lin} = \unit{0.18}{\ohm\per\meter}$) :

$$R = L_{fil} \cdot R_{lin} = \unit{7.6}{\ohm}$$

\begin{table}[!htb]
	\centering
	\begin{tabular}{c|c|c|c}
		$N_1$ & $B_e$ & $R$ & $L_{fil}$ \\
		\hline
		420 & $\unit{0.07539825}{\tesla}$ & \unit{7.6}{\ohm} &  $\unit{42.22}{\meter}$\\
	\end{tabular}
	\caption{Tableau récapitulatif pour l'électroaimant.}
\end{table}

\subsection{Calcul de la constante de raideur de la membrane}
Avant de pouvoir déterminer le nombre de spires de la bobine mouvante, nous avons dû déterminer
expérimentalement la constante de raideur de notre assemblage papier-tissus utilisé pour faire la membrane.
Notre procédure a été la suivante : nous avons suspendu notre membrane, pour ensuite 
y déposer différentes masses, et finalement mesurer son élongation.
Nous obtenons ainsi une constante de raideur d'environ \unit{85}{\newton\per\meter}.

\subsection{Fonctionnement et dimensionnement de la bobine mobile}
La bobine mobile va intercepter le flux situé dans l'entrefer de l'életroaimant (Figure 
\ref{overview_mobile_coil}).
Comme cette bobine mobile est traversé par un courant $i(t)$ qui dépend du signal
audio envoyé par le circuit du haut-parleur, elle va subir une force de \textsc{Laplace}
dont l'expression est :

$$\vec{F} = i(t)\vec{L}\times{\vec{B}}$$ 

Dans cette sous-section, nous allons calculer le nombre de spires nécessaires
à cette bobine mobile afin qu'elle subisse une force capable de la faire
se déplacer de \unit{2}{\milli\meter} (écartement maximal de la membrane 
par rapport à sa position d'origine, choisi arbitrairement).

\begin{figure}[ht!]
\centering
\includegraphics[scale=0.3]{hautparleur.png}
\caption{Vue d'ensemble avec la seconde bobine.}
\label{overview_mobile_coil}
\end{figure}

\paragraph{Calcul du nombre de spires}
Etant donné que nous disposons d'un amplificateur qui, selon la datasheet\cite{datasheetampli}, a une puissance de sortie de 
$\unit{2.5}{\watt}$, et que la tension de sortie est de $\unit{15}{\volt}$ (en valeur efficace), nous pouvons trouver le courant
maximal passant dans la bobine mobile :

$$I = \frac{P}{V} \cdot{\sqrt{2}} = \unit{0.2357}{\ampere}$$

En fonction de la constante de raideur de la membrane trouvée dans la sous-section précédente et de l'écartement
maximal de la membrane par rapport à sa position d'origine (fixé à $d = \unit{2}{\milli\meter}$), nous sommes en
mesure de trouver la longueur du fil de la bobine:

$$IL_{fil}B = kx \Rightarrow L_{fil} = \frac{kx}{IB} = \unit{9.57}{\meter}$$

En fixant le rayon $r=\unit{17}{\milli\meter}$, nous pouvons déterminer $N_2$ :

$$L_{fil} = N_2 \cdot 2\pi r \Rightarrow N_2 =  \frac{L_{fil}}{2\pi r} = 89.6$$

Ici aussi, nous avons fait l'hypothèse que les spires constituant la bobine forment des cercles parfaits.
En connaissant l'encombrement de la bobine, qui est de \numprint{25.8} spires par\unit{}{\centi\meter}, nous
pouvons également calculer la longueur de la bobine :

$$L_{bobine} = \frac{N_2}{25.8} = \unit{3.48}{\centi\meter}$$

\paragraph{Calcul de la résistance totale de la bobine mobile}
Pour calculer la résistance totale de la bobine, il ne nous reste plus qu'à multiplier la longueur de fil trouvée 
précédemment par la résistance linéique du fil de cuivre
($R_{lin} = \unit{0.18}{\ohm\per\meter}$) :

$$R = L_{fil} \cdot R_{lin} = \unit{1.72}{\ohm}$$

\begin{table}[!htb]
	\centering
	\begin{tabular}{c|c|c|c|c}
		$N_2$ & $I$ & $R$ & $L_{fil}$ & $L_{bobine}$\\
		\hline
		 $90$ & $\unit{0.2357}{\ampere}$ & $\unit{1.72}{\ohm}$ & \unit{9.57}{\meter} & \unit{3.48}{\centi\meter}
	\end{tabular}
	\caption{Tableau récapitulatif pour la bobine mobile}
\end{table}

% Just here to fix rapport_prejury.tex
\end{document}