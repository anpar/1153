\documentclass{article}

\usepackage[utf8]{inputenc}
\usepackage[T1]{fontenc}      
\usepackage[francais]{babel}
\usepackage{graphicx}
\usepackage{circuitikz}
\usepackage[squaren, Gray]{SIunits}
\usepackage{sistyle}
\usepackage[autolanguage]{numprint}
\usepackage{pgfplots}
\usepackage{amsmath,amssymb,array}
\usepackage{url} 

% New command pour la modélisation mécanique, tri à effectuer
\newcommand\fv[1]{{\bf #1}} % free vector
\newcommand\fvd[1]{\dot{\bf #1}} % free vector derivated
\newcommand\fvdd[1]{\ddot{\bf #1}} % free vector derivated
\newcommand\fvr[1]{\mathring{\bf #1}} % free vector relatively derivated
\newcommand\fvrr[1]{\overset{\circ\circ}{\bf #1}} % free vector relatively derivated
\newcommand\uv[1]{{\bf\hat{ #1}}} % unit vector
\newcommand\ui{{\bf\hat{I}}} % unit vector I
\newcommand\uj{{\bf\hat{J}}} % unit vector J
\newcommand\uk{{\bf\hat{K}}} % unit vector K
\newcommand\wrt[2]{\ensuremath{\tensor*[_{ #1}]{ #2}{}}} % With Respect To
\newcommand\wtr[3]{\ensuremath{\tensor*[_{ #1}]{ #2}{^{ #3}}}} % With Two Respect
\newcommand\omegaf{{\bm \omega}}
\newcommand\omegafr{\mathring{\bm \omega}}
\newcommand\omegafd{\dot{\bm \omega}}
\newcommand\omegaft{\tilde{\bm \omega}}
\newcommand\omegaftr{\mathring{\tilde{\bm \omega}}}
\newcommand\omegat{\tilde{\omega}}
\newcommand\omegatd{\tilde{\dot{\omega}}}
\newcommand\ine{{\bf I}}
\newcommand\st{{\bf L}}
\newcommand\pst{{\bf M}}
\newcommand\lm{{\bf N}}
\newcommand\am{{\bf H}}
\newcommand\amd{\dot{\am}}
\newcommand\fo{{\bf F}}
\newcommand\po{\mathcal{P}}
\newcommand\xg{\ensuremath{\fv{R}}}
\newcommand\xgd{\ensuremath{\fvd{R}}}
\newcommand\xgdd{\ensuremath{\fvdd{R}}}
\newcommand\dvec[1]{\dot{\vec{ #1}}}
\newcommand\ddvec[1]{\ddot{\vec{ #1}}}
\newcommand\qp{\dot{q}}
\newcommand\dqp{\Delta \dot{q}}

\begin{document}


\section{Modélisation mécanique du haut-parleur}

\subsection{Composition du haut-parleur}
Le haut-parleur est constitué de différentes parties : une membrane
attachée par des fixations qui jouent le rôle de ressorts et reliée à 
une bobine mobile et une bobine fixe qui s'emboîte (sans frottement) 
dans cette dernière.

La bobine fixe va permettre à la bobine mobile de se déplacer exclusivement de gauche
à droite (voir Figure \ref{hp-scheme}), permettant ainsi à la membrane de vibrer 
(et donc de produire un son). Tout déplacement dans une autre direction serait dommageable
car risquerait d'abîmer la membrane.

\begin{figure}[ht!]
	\centering
	\includegraphics[scale=0.6]{haut-parleur.png}
	\caption{Schéma d'un haut-parleur. (Source : \textit{Modélisation mécanique du haut-parleur} sur iCampus)}
	\label{hp-scheme}
\end{figure}

\subsection{Étude du mouvement de la bobine mobile}
Nous allons maintenant écrire les équations du mouvement de la bobine mobile.
Pour ce faire, nous plaçons un repère fixe $\{\ui\}$ dont l'origine $O$ se trouve
au centre de gravité de la bobine mobile à sa position d'équilibre (au 
temps $t=0$). $\uv{I}_1$ est parallèle à la bobine et dirigé vers la gauche, tandis que
$\uv{I}_2$ est dirigé perpendiculairement à la bobine mobile, vers le haut.
La bobine mobile ne possède qu'un seul degré de liberté, qui
est la distance entre $O$ et son centre de gravité; notons-la $\fv{x}(t)$.
La positon de la bobine est donc donnée par :

$$\vec{R(t)} = \fv{x}(t) \uv{I}_1$$ 

\paragraph{Inventaire des forces}
Avant d'écrire l'équation du mouvement de la bobine, établissons l'inventaire
des forces qui agissent sur celle-ci :

\begin{itemize}
	\item Son poids, dont la résultante agit sur son centre de gravité : $-mg\uv{I}_2$ ;
	\item La force de rappel des fixations (que l'on suppose agir comme des simples
	ressorts) : $-k \fv{x}(t) \uv{I}_1$ où $k$ est la constante de raideur des fixations ;
	\item La force électromagnétique causée par l'électroaimant : $BLi(t) \uv{I}_1$ où
	$B$ est le champ magnétique produit par l'électroaimant, $L$ la longueur de fil de cuivre
	utilisée et $i(t)$ le courant électrique ;
	\item Le frottement dû à l'air ;
\end{itemize}

Parmis toute ces forces, nous négligeons le frottement dû à l'air ainsi que le poids
de la bobine mobile (sa masse étant relativement faible).

\paragraph{Équation du mouvement}
Nous avons maintenant tout à notre disposition pour écrire les équations du mouvement
\footnote{Dans cette section, nous utilisons les notations employées au cours de
mécanique des corps rigides.} :

$$m\fvdd{x}(t) = -k\fv{x}(t) + BLi(t)$$

En sachant que le signal d'entrée est une fonction de la forme $V(t) = V_0 \cos (\omega t)$ et
que, par la loi d'Ohm, $V(t) = Ri(t)$ où $R$ est la résistance du circuit, 
nous pouvons réecrire l'équation différentielle du mouvement de la manière suivante :

$$m\fvdd{x}(t) + k\fv{x}(t) = \frac{B2\pi rNV_0}{R}\cos (\omega t)$$

Où nous avons également fait apparaître le nombre de spires $N$ et le rayon de la bobine
$r$. Il ne reste donc plus qu'à résoudre cette équation différentielle.

\paragraph{Résolution de l'équation différentielle du mouvement}
Résolvons cette équation différentielle comme appris lors de ce deuxième
quadrimestre. Cherchons d'abord la solution homogène de cette équation, notée $\fv{x}_h(t)$.
Pour ce faire, résolvons le polynôme caractéristique :

$$mr^2 + k = 0 \Rightarrow r = \pm i\sqrt{\frac{k}{m}}$$

Nous avons donc, en ne gardant que la partie réelle :

$$\fv{x}_h(t) = Ae^{i\sqrt{\frac{k}{m}}t} + Be^{-i\sqrt{\frac{k}{m}}}$$

Où $A$ et $B$ sont des coéfficients complexes. Nous pouvons réecrire cette solution
en termes de fonctions trigonométriques. En ne gardant que la partie réelle, 
nous obtenons :

$$\fv{x}_h(t) = C\cos(\sqrt{\frac{k}{m}}t) - D\sin(\sqrt{\frac{k}{m}}t)$$

Où $C$ et $D$ sont cette fois des coefficients réels.

Penchons-nous maintenant sur la solution particulière, notée $\fv{x}_p(t)$. Pour
cette partie de la résolution, nous réécrivons le terme non-homogène sous la forme d'une
exponentielle complexe. La solution particulière est de la forme :

$$\fv{x}_p(t) = \alpha e^{\omega it}$$

En injectant $\fv{x}_p(t)$ et sa dérivée seconde dans l'équation de départ, nous trouvons :

$$\alpha = \frac{2\pi rNV_0}{R(-m\omega^2 + k)} \Rightarrow \fv{x}_p(t) = \frac{2\pi rNV_0}{R(-m\omega^2 + k)}e^{wit}$$

En ne gardant que la partie réelle de l'exponentielle, nous avons finalement :

$$\fv{x}_p(t) = \frac{2\pi rNV_0}{R(-m\omega^2 + k)} \cos (\omega t)$$

Par le principe de superposition des solutions des équations différentielles :

$$\fv{x}(t) = \fv{x}_h(t) + \fv{x}_p(t) = C\cos(\sqrt{\frac{k}{m}}t) - D\sin(\sqrt{\frac{k}{m}}t) + \frac{2\pi rNV_0}{R(-m\omega^2 + k)} \cos (\omega t)$$

Pour trouver l'équation finale sans les coéfficients inconnus $C$ et $D$, il faudrait utiliser deux conditions
initiales. La première condition initiale est assez simple à trouver. En effet, au temps $t=0$, la bobine
mobile n'a pas encore commencé à bouger, nous pouvons donc écrire :

$$\fv{x}(0) = 0$$

Ce qui nous donne immédiatement $C$. La deuxième condition initiale pourrait par exemple impliquer le fait
que la vitesse de la bobine est nulle lorsqu'elle atteint son déplacement maximal (donc juste avant qu'elle 
commencer à retourner vers sa position d'équilibre). Cependant pour écrire cette condition initiale, il faudrait
connaître le temps $t_f$ que mets la bobine pour atteindre cette position maximale. On pourrait alors écrire :

$$\fvd{x}(t_f) = 0$$

Nous ne connaissons malheureusement pas ce temps qu'il faudrait mesurer expérimentalement.

\subsection{Fréquence de résonance}
Nous remarquons que si $\omega \rightarrow \sqrt{\frac{k}{m}}$, le dénominateur d'un des termes de l'équation du mouvement
tend vers \numprint{0}, et donc l'amplitude du mouvement tend vers $\infty$. Cette situation n'a physiquement
pas de sens. Nous notons cette fréquence $\omega_0$; il s'agit de la \textit{fréquence de résonance}.

% Il faudrait peut-être refaire un lien avec les autres endroits ou on parle de fréquence de résonance?
% Ou bien avec la partie dimensionnement du haut-parleur et de la membrane ?

% Just here to fix rapport_prejury.tex
\end{document}