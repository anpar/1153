\input{../head.tex}

\section{Modélisation mécanique du haut-parleur}
%explication intuitive du fonctionnement
Commençons par un bref rappel des différents constituants du haut-parleur:
un boîtier, une membrane attachée par des fixations jouant le rôle de ressorts, 
un électroaimant et une bobine mobile qui s'emboîte (sans frottement)
dans ce dernier.

L'électroaimant constitué d'une bobine fixe va permettre à la bobine mobile de se déplacer exclusivement de gauche
à droite, permettant ainsi à la membrane de vibrer (et donc de produire un son). Tout déplacement dans une autre direction serait dommageable car risquerait d'abîmer la membrane\(cfr Figure \ref{hp_scheme).

%schémas et figures supportant le développement théorique
\begin{figure}[ht!]
\centering
\includegraphics[scale=0.6]{haut-parleur.png}
\caption{Schéma d'un haut-parleur. (Source : \textit{Modélisation mécanique du haut-parleur} sur iCampus)}
\label{hp-scheme}
\end{figure}
\subsubsection{Étude du mouvement de la bobine mobile}
%previsions
Nous allons maintenant écrire les équations du mouvement de la bobine mobile. Cela nous amènera à un cas 
particulier: la résonance. Ce phénomène apparaît lorsque la fréquence de la force exercée sur la bobine 
mobile devient très proche de la fréquence propre\footnote{fréquence naturelle, sans force extérieure} 
du système\cite{resonance}. Nous arrivons à une amplitude maximale, et le son produit sera plus intense.
Dans cette section, nous nous pencherons sur l'équation du mouvement de la bobine mobile, et nous en
déduirons la fréquence de résonance de notre haut-parleur. Nous vérifierons que cette fréquence particulière
est également la fréquence propre de notre dispositif.


Pour commencer, plaçons un repère fixe $\{\hat{I}\}$ dont l'origine $O$ se trouve
au centre de gravité de la bobine mobile à sa position d'équilibre (au
temps $t=0$). $\hat{I}_1$ est parallèle à la bobine et dirigé vers la gauche, tandis que
$\hat{I}_2$ est dirigé perpendiculairement à la bobine mobile, vers le haut.
La bobine mobile ne possède qu'un seul degré de liberté, qui
est la distance entre $O$ et son centre de gravité; notons-la $\fv{x}(t)$.
La position de la bobine est donc donnée par :

$$\vec{R(t)} = \fv{x}(t) \hat{I}_1$$

\paragraph{Inventaire des forces}
Avant d'écrire l'équation du mouvement de la bobine, établissons l'inventaire
des forces qui agissent sur celle-ci :

\begin{itemize}
\item Son poids, dont la résultante agit sur son centre de gravité : $-mg\hat{I}_2$ ;
\item La force de rappel des fixations (que l'on suppose agir comme des simples
ressorts) : $-k \fv{x}(t) \hat{I}_1$ où $k$ est la constante de raideur des fixations ;
\item La force électromagnétique causée par l'électroaimant : $BLi(t) \hat{I}_1$ où
$B$ est le champ magnétique produit par l'électroaimant, $L$ la longueur de fil de cuivre
utilisée et $i(t)$ le courant électrique ;
\item Les frottements ;
\end{itemize}
%hypothèses de modélisation
Parmi toute ces forces, nous négligeons les frottements ainsi que le poids
de la bobine mobile (sa masse étant relativement faible).
%formalisme de modélisation
\paragraph{Équation du mouvement}
Nous avons maintenant tout à notre disposition pour écrire les équations du mouvement
\footnote{Dans cette section, nous utilisons les notations employées au cours de
mécanique des corps rigides.} :

$$m\fvdd{x}(t) = -k\fv{x}(t) + BLi(t)$$

En sachant que le signal d'entrée est une fonction de la forme $V(t) = V_0 \cos (\omega t)$ et
que, par la loi d'Ohm, $V(t) = Ri(t)$ où $R$ est la résistance du circuit,
nous pouvons réécrire l'équation différentielle du mouvement de la manière suivante :

$$m\fvdd{x}(t) + k\fv{x}(t) = \frac{B2\pi rNV_0}{R}\cos (\omega t)$$

Où nous avons également fait apparaître le nombre de spires $N$ et le rayon de la bobine
$r$. Il ne reste donc plus qu'à résoudre cette équation différentielle.

\paragraph{Résolution de l'équation différentielle du mouvement}
Résolvons cette équation différentielle comme appris lors de ce deuxième
quadrimestre. Cherchons d'abord la solution homogène de cette équation, notée $\fv{x}_h(t)$.
Pour ce faire, résolvons le polynôme caractéristique :

$$mr^2 + k = 0 \Rightarrow r = \pm i\sqrt{\frac{k}{m}}$$

Nous avons donc, en ne gardant que la partie réelle :

$$\fv{x}_h(t) = Ae^{i\sqrt{\frac{k}{m}}t} + Be^{-i\sqrt{\frac{k}{m}}}$$

Où $A$ et $B$ sont des coefficients complexes. Nous pouvons réécrire cette solution
en terme de fonctions trigonométriques. En ne gardant que la partie réelle,
nous obtenons :

$$\fv{x}_h(t) = C\cos(\sqrt{\frac{k}{m}}t) - D\sin(\sqrt{\frac{k}{m}}t)$$

Où $C$ et $D$ sont cette fois des coefficients réels.

Penchons-nous maintenant sur la solution particulière, notée $\fv{x}_p(t)$. Pour
cette partie de la résolution, nous réécrivons le terme non-homogène sous la forme d'une
exponentielle complexe. La solution particulière est de la forme :

$$\fv{x}_p(t) = \alpha e^{\omega it}}$$

En injectant $\fv{x}_p(t)$ et sa dérivée seconde dans l'équation de départ, nous trouvons :

$$\alpha = \frac{2\pi rNV_0}{R(-m\omega^2 + k)} \Rightarrow \fv{x}_p(t) = \frac{2\pi rNV_0}{R(-m\omega^2 + k)}e^{wit}$$

En ne gardant que la partie réelle de l'exponentielle, nous avons finalement :

$$\fv{x}_p(t) = \frac{2\pi rNV_0}{R(-m\omega^2 + k)} \cos (\omega t)$$

Par le principe de superposition des solutions des équations différentielles :

$$\fv{x}(t) = \fv{x}_h(t) + \fv{x}_p(t) = C\cos(\sqrt{\frac{k}{m}}t) - D\sin(\sqrt{\frac{k}{m}}t) + \frac{2\pi rNV_0}{R(-m\omega^2 + k)} \cos (\omega t)$$

En utilisant la première condition initiale, $\fv{x}(0) = 0$, nous trouvons:

$$C = \frac{-BLV_0}{R(-m\omega^2+k}$$



\subsubsection{Fréquence de résonance}
%prévisions en fct des paramètres
Nous remarquons que si $\omega \rightarrow \sqrt{\frac{k}{m}}$, le dénominateur de l'un des termes de l'équation du mouvement
tend vers \numprint{0}, et donc l'amplitude du mouvement tend vers $\infty$. Cette situation n'a physiquement
pas de sens: le déplacement de la partie mobile serait infini. Nous notons cette fréquence $f_0 = \frac{\omega_0}{2\pi}$; il s'agit de la \textit{fréquence de résonance}.
Cette fréquence est également la fréquence propre de notre dispositif: nous pouvons identifier cet 
appareillage à une masse (la bobine) reliée à un ressort et se mouvant dans une direction. La formule
de la fréquence d'oscillation d'un tel système correspond exactement à notre fréquence de 
résonance\cite{resonance}. Pour une masse de \SI{0.02}{\kilogram} et une constante de rigidité 
de \SI{85}{\frac{\newton}{\meter}}, nous obtenons une fréquence de résonance pour notre haut-parleur 
de: \SI{10.38}{\hertz}

% Il faudrait peut-être refaire un lien avec les autres endroits ou on parle de fréquence de résonance?
% Ou bien avec la partie dimensionnement du haut-parleur et de la membrane ?

\input{../foot.tex}
