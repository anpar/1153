\documentclass{article}

\usepackage[latin1]{inputenc}
\usepackage[T1]{fontenc}
\usepackage[francais]{babel}
\usepackage{graphicx}

\begin{document}

\section{Dimmensionnement de l'électroaimant fixe}


Etant donné que nous ne disposons pas d'aimant permanent, nous avons dû fabriquer un électroaimant. Nous avons donc bobiné un fil de cuivre autour du matériau ferromagnétique qui nous avait été fourni. 


\begin{figure}[h]
\centering
\includegraphics[scale=0.6]{electroaimant.png}
\caption{Modélisation d'un électroaimant}
\label{modélisation de l'électroaimant}
\end{figure}



\paragraph{Fonctionnement}
Lorsqu'un courant traverse la bobine de cuivre, un champ magnétique est formé.  Nous obtenons donc un électroaimant fixe générant le champ nécessaire au déplacement de la seconde bobine. C'est cette seconde bobine qui sera responsable du tremblement de la membrane.

\begin{figure}[h]
\centering
\includegraphics[scale=0.3]{hautparleur.png}
\caption{Vue d'ensemble avec la seconde bobine}
\label{Vue d'ensemble avec la seconde bobine}
\end{figure}

\paragraph{Calul du nombre de spires}

Dans cete situation, la majorité du champ magnétisant se retrouve dans l'"`entrefer"' de 11mm. En supposant qu'il n'y a pas de perte de flux, et que le champ magnétique B vaut  ... T, nous sommes en mesure de calculer le nombre de spires nécessaires pour faire bouger la membrane de 3mm.
En utilisant les formules suivantes:

$$\oint \vec{H}\cdot\vec{dl}\cong H_m * L + H_e * e = N*I$$ \\
et:
$$B_e = \mu_0*H_e$$\\

Nous obtenons finalement ... spires.



%  ENCORE A DETERMINER

%* nbre de spires
%* éloigné?
%* longueur de la bobine
%* aire du matériau ferromagnétique (combien de plaques)



\end{document}
