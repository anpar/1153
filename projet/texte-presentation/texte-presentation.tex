\documentclass{article}

\usepackage[utf8]{inputenc}
\usepackage[T1]{fontenc}      
\usepackage[francais]{babel}
\usepackage[squaren, Gray]{SIunits}
\usepackage{sistyle}
\usepackage[autolanguage]{numprint}
\usepackage{amsmath,amssymb,array}
\usepackage[top=2.5cm,bottom=2.5cm,right=2.5cm,left=2.5cm]{geometry}
\usepackage{color}

\title{Feuille de route défense orale}
\author{Groupe 1153}
\date{\today}

\begin{document}
\maketitle

\section{Introduction (Virgile)}
Chers membres du jury bonjour,
Selon notre cahier des charges nous devions créer un haut-parleur capable d'émettre un son,
l'amplifié et faire varier les basses et aigus, la fréquence audible du son devait se trouver
entre \unit{500}{\hertz} et \unit{5000}{\hertz} et délivré une puissance de \unit{2.5}{\watt},
à partir d'une prise jack de \unit{3.5}{\milli\meter}, une tension d'alimentation de $\unit{\pm 15}{\volt}$ 
et une membrane de papier. C'est ce travail que le groupe 115.3 va vous présenter aujourd'hui.
Tout d'abord Thibaut va vous présenter le fonctionnement général du haut parleur, Lise
la modélisation des filtres, je ferai ensuite le dimensionnement de l'électroaiment et de la bobine mobile, 
Marie-Charlotte s'occupera du dimensionnement de la membrane et du haut parleur. Antoine s'occupera quant à lui
de la modélisation mécanique de la bobine mobile et de la contre réaction. Enfin, Robin vous parlera de la
distortion harmonique et je finirai avec la conclusion.

\section{Fonctionnement général (Thibaut)}

\textcolor{blue}{J'ai remis le texte que Marie-Charlotte avait fait comme c'était sa partie avant.}

Le GSM ou le MP3 va dans un premier temps envoyer un signal audio via le câble Jack au circuit imprimé. Celui-ci permet
de modifier ce signal brut de plusieurs façons :

\begin{itemize}
	\item En réglant le volume, c'est-à-dire en modifiant l'amplitude du signal audio ;
	\item En réglant les graves et les aigus grâce au filtre passe-bande ;
	\item En amplifiant le signal : c'est le rôle de l'amplificateur audio du circuit.
\end{itemize}

À la sortie du circuit imprimé, le signal est alors \textit{filtré} et \textit{amplifié}.
Ce signal traité ira ensuite alimenter en courant la bobine mobile. Cette dernière intercepte un champ magnétique constant 
produit par un électroaimant formé d'une bobine de fil de cuivre, d'un matériau ferromagnétique, et alimenté en courant continu.
La bobine mobile subit donc une force de \textsc{Laplace}, va osciller en fonction du  signal audio, et reproduira le son voulu.
Enfin, la membrane pourra revenir à sa position d'équilibregrâce à des attaches qui, à la manière de ressorts, produisent une
force de rappel dans la direction opposée au mouvement de la bobine mobile. 

\section{Modélisation des filtres (Lise)}

\textcolor{blue}{A faire...}

\section{Dimensionnement électroaimant et bobine fixe (Virgile)}

\textcolor{blue}{J'ai rajouté les remarques qu'on avait faite tantôt : les courants de Foucaults, la forme de l'électroaimant, etc.}

Pour fabriquer notre haut-parleur, nous ne disposons pas d'aimant permanent. Nous avons du créer un électroaimant à partir d'un
matériau ferromagnétique qui nous a été fourni. Deuc choses sont à notée à propos de cet électroaimant :

\begin{itemize}
	\item Il est constitué de fines lamelles empilées les unes sur les autres, et ce afin de réduire les courants de \textsc{Foucault} ;
	\item La forme de l'électroaimant (en "E") permet de concentrer tous le champ dans l'entrefer.
\end{itemize}

 Cette section présente dans un premier temps le dimensionnement de l'électroaimant, c'est à dire le nombre 
de spires choisies, la résistance totale de la bobine...
Le nombre de spires total de la bobine fixe a été choisi de manière arbitraire, sur les conseils de notre tuteur à 420. Pour ce nombre de spires, un entrefer 
réduit à \unit{7}{\milli\meter} et un courant de \unit{1}{\ampere}, nous obtenons une champ magnétique de \unit{75}{\milli\tesla}.
Avec 420 spires, nous obtenons une longueur de fil de \unit{42.22}{\meter} ce qui avec notre résistance linéique de 0.18$\Omega$ par mètre
nous donne une résistance totale de 7.6$\Omega$ .
Ensuite la bobine mobile va intercepter le champ magnétique de la bobine fixe et va donc subir une force de \textsc{Laplace} qui va lui permettre de faire 
bouger la membrane, dans notre cas de \unit{2}{\milli\meter}. Il faut donc que la force magnétique soit égal à la force de rappel,
$kx$. Pour obtenir cette force il nous faut un fil long de \unit{9.57}{\meter} donc 90 spires pour une résistance totale de 1.72$\Omega$.

\section{Dimensionnement de la membrane et du haut-parleur (Marie-Charlotte)}
\subsection{Le boîtier}
La première question qui s'est posée était celle du volume du caisson. Nous nous sommes renseignés sur la fabrication de 
haut-parleurs, et nous avonsappris que le volume est lié à de nombreux paramètres, et qu'un volume trop petit ne restituerait 
pas bien les extrèmes graves. Etant donné que nous devions pouvoir faire varier la fréquence sur notre dispositif, nous avons 
finalement opté pour un boîtier cubique de $\unit{25}{\centi\meter}$ de côté.Afin d'améliorer un peu le boîtier, nous avons
également pensé à placer des pieds en caoutchouc afin de réduire les déplacements dûs aux vibrations du haut-parleur.Enfin,
pour faciliter l'accès à l'intérieur du haut-parleur, nous avons créé un système de porte coulissante à l'arrière de celui-ci.

\subsection{La membrane}
Nous avons opté pour une membrane de diamètre de $\unit{17}{\centi\meter}$, pour exploiter le mieux possible la taille du
caisson. C'est également undiamètre assez répandu dans le commerce. La profondeur correspondante est de $\unit{6}{\centi\meter}$.
Elle est réalisée en papier, et nous avons opté pour du tissus tendu en guise de ressort. Cette solution nous apparaît comme
sortant de l'ordinaire, propre, et efficace. Cela nous a en effet permis d'obtenir une constante de raideur minime.

\section{Modélisation mécanique de la bobine mobile (Antoine)}
Après avoir fabriquer l'assemblage membrane/bobine mobile, nous nous sommes pencher
sur l'équation du mouvement de cette bobine mobile. Pour ce faire, nous avons appliqué
ce que nous avons appris au cours de mécanique des corps rigides tout au long de ce quadrimestre.
Nous avons donc commencé par faire le bilan des forces agissant sur la bobine, il y en a 4 en tout :

\begin{itemize}
	\item Le poids de la bobine ;
	\item La force de \textsc{Laplace} subie par la bobine ;
	\item La force de rappel produit par le tissu utilisé ;
	\item Le frottement dû à l'air.
\end{itemize}

Parmi toutes ces forces, nous négligeons le frottement dû à l'air et le poids de la bobine.
Notre équation de base est donc celle indiquée sur le slide. La résolution de cette équation
nous permet de remarque quelque chose d'important : quand la fréquence d'oscillation
de la bobine tend vers $\frac{1}{2\pi} \sqrt{\frac{k}{m}}$ (qui est appelée la fréquence de
résonance), l'amplitude du mouvement de la bobine tend vers l'infini. Cette situation est donc
dommageable pour notre haut-parleur (car le membrane pourrait s'abîmer). Pour notre haut parleur,
la fréquence de résonance est de l'ordre de \unit{10}{\hertz}.

\textcolor{blue}{Notes à moi même : 
\begin{itemize}
	\item Un système résonant peut accumuler une énergie, si celle-ci est appliquée sous
				forme périodique, et proche d'une fréquence dite « fréquence de résonance ». Soumis à une telle
				excitation, le système va être le siège d'oscillations de plus en plus importantes, jusqu'à 
				atteindre un régime d'équilibre qui dépend des éléments dissipatifs du système, ou bien jusqu'à 
				une rupture d'un composant du système ;
	\item Si on veut descendre bas dans le grave, il faut une fréquence de résonance basse.
\end{itemize}}

\section{Contre-réaction (Antoine)}
Passons maintenant à la recherche documentaire. Pour cette recherche, nous avons choisi deux 
sujets qui, comme vous le verrez sont un peu liés. Il s'agit de la distortion harmonique, dont vous
parlera Robin juste après, et du principe de contre-réaction.

Définissons d'abord ce qu'est le principe de réaction en général. La réaction consiste en la réinjection
d'une partie du signal de sortie à l'entrée. Plus spécifiquement, dans la contre-réaction (aussi appelée 
réaction négative), le signal réinjecté est en opposition de phase avec la signal d'entrée. Les effets
de la contre-réaction sont multiples :

\begin{itemize}
	\item Signal de sortie plus proche du signal d'entrée ;
	\item	Réduction des signaux parasites et de la distorsion ;
	\item \textbf{Contrôle du gain} ;
	\item Elargissement de la bande passante ;
	\item Réduction de l'impédance de sortie.
\end{itemize}

Dans le cas particulier du circuit de notre haut-parleur, l'effet principal de la boucle de contre-réaction
est de ramener le gain de l'amplificateur à 1 (tension de sortie $=$ tension d'entrée). Un tel amplificateur
est appelé un amplificateur suiveur de tension (impédance d'entrée infinie, donc courant d'entrée nul).

\section{Distorsion harmonique (Robin)}

La distorsion, par défininition, est une transformation du signal de sortie par rapport à celui
d'entrée. Etant donné que le signal en est déformé, elle n'est pas souhaitée.
Nous allons vous parler d'un type particulier de distorsion: la distorsion harmonique.
La distorsion harmonique joue sur la supperposition de différentes fréquences: une fondamentale 
et ses harmoniques, qui sont parasites. 

Parlons de ses causes. C'est l'appareil lui-même qui crée la distorsion. Dans notre circuit, il 
s'agit de notre ampli-audio. Les principales causes sont les charges non-linéaires, une charge étant 
un élément de circuit qui consomme une puissance. Elles apparaissent lorsque la résistance n'est pas constante, 
et  cause la création des courants harmoniques, responsables de la distorsion harmonique. 

Cette distorsion a pour conséquence un accroissement de courant qui se caractérise par un vieillisement 
préaturé des compisantes électriques.

Plusieurs moyens sont mis en oeuvre pour l'éviter. C'est le cas des filtres actifs. Ils injectent des fréquences 
identiques aux courants harmoniques mais déphasés d'une demi période. Il en résulte une interférence destructrice 
qui ne change pas le signal d'aentrée. La boucle de conre-réaction, est aussi un moyen de diminuer de manière significative
le TDH, taux de distorsion harmonique.

\section{Planning (Thibaut)}

\begin{itemize}
	\item Récapitulatif des choses réalisées (comme on peut le voir, nous avons fait…) ;
	\item Première analyse : échéances respectées, bonne organisation (contrairement Q1) ;
	\item Deuxième analyse : beaucoup de théorie au début- pratique à la fin  (comprendre puis agir) ;
	\item Troisième analyse : rédaction du rapport tout le long du quadrimestre (Guithub, déjà préjury).
\end{itemize}

\section{Conclusion (Virgile)}
\begin{itemize}
    \item Nos erreurs ;
    \item Nos apprentissages ;
    \item Notre ressenti.
\end{itemize}
		
\end{document}