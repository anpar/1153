\documentclass{article}

\usepackage[utf8]{inputenc}
\usepackage[T1]{fontenc}      
\usepackage[francais]{babel}
\usepackage[squaren, Gray]{SIunits}
\usepackage{sistyle}
\usepackage[autolanguage]{numprint}
\usepackage{amsmath,amssymb,array}
\usepackage[top=2.5cm,bottom=2.5cm,right=2.5cm,left=2.5cm]{geometry}

\begin{document}
\section{Thibaut}
\subsection{Planning}

\begin{itemize}
	\item Récapitulatif des choses réalisées (comme on peut le voir, nous avons fait…) ;
	\item Première analyse : échéances respectées, bonne organisation (contrairement Q1) ;
	\item Deuxième analyse : beaucoup de théorie au début- pratique à la fin  (comprendre puis agir) ;
	\item Troisième analyse : rédaction du rapport tout le long du quadrimestre (Guithub, déjà préjury).
\end{itemize}

\subsection{Validation}
Pour savoir faire le haut-parleur : données pour nombre de tour de bobinage,…  A faire avec appareil dont nous ne connaissions rien avant.
Présentation de chacun et utilité :
\begin{itemize}
	\item Source de tension ;
	\item Oscilloscope ;
	\item Multimètre ;
	\item Teslamètre.
\end{itemize}

Compte-rendu : données 
Analyse : avec pento + faible que sans pento…  Erreur quelque part

\section{Lise}
\subsection{Approximation mathématique}

Dans le cadre de notre projet, il nous a été demandé de réaliser une approximation mathématique de la fréquence de coupure
du le filtre passe-bas. C'est à dire, calculer l'intersection entre une droite constante représentant la tension du circuit à basse fréquence et une droite représentant la tension du circuit à hautes fréquences. 

Tout d'abord, graphiquement, cette droite dans un repère cartésien représente un logarithme. Nous sommes donc passé en échelle semi-logarithmique pour avoir une droite oblique.

Ensuite, à l'aide des données prisent en laboratoire, nous devions déterminer l'équation de cette droite. Trois méthodes permettant de minimiser la distance entre deux droites nous ont été proposé. Nous avons décidé d'utiliser
la méthode de la solution approchée à l'aide de la projection orthogonale sur une base orthonormée du système. 

Nous avons ensuite posé le système suivant : $y = a log(x) + b$ où y représente la tension de sortie du filtre passe-bas et x la fréquence. Nous cherchons à déterminer les valeurs a et b, 
respectivement la pente et l'ordonnée à l'origine de la droite. Les données prise en laboratoire n'étant pas précise, nous arrivons à un système impossible. 

Nous avons orthonormée la base des colonnes du système pour ensuite  déterminer une solution approché à l'aide de la projection orthogonale. 

Nous avons ainsi trouver les valeurs pour les coefficients a et b et ensuite déterminer l'intersection avec la droite 2.5V.
Nous obtenons ainsi une fréquence de coupure de 5557,7 Hz. 

Cette valeur est fort proche de la valeur théorique $f = \frac{1}{2\pi RC}$ qui est égale dans notre cas à 5889,2 Hz.

\section{Robin}
% TODO

\section{Virgile}
\subsection{Introduction}
Chèrs membres du jury bonjour,
Selon notre cahier des charges nous devions créer un haut-parleur capable d'émettre un son,
l'amplifié et faire varier les basses et aigus, la fréquence audible du son devait se trouver entre 500 et 5000Hz
et délivré une puissance de 2.5W, à partir d'une prise jack de 3.5mm, un voltage de 30V et une membranne de papier.
C'est ce travail que le groupe 11.53 va vous présenter aujourd'hui.
Tout d'abord thibaut va vous présenter le fonctionnement général du haut parleur,
je ferai le dimensionnement de l'électroaiment, ??? s'occupera du dimensionnement du haut parleur,
??? de l'aproximation mathématique, ??? de la validation, ??? de la modélisation des filtres,
??? de la recherche documentaire et je finirai avec la conclusion.
\subsection{Dimensionnement électroaimant}
Pour fabriquer notre haut-parleur, nous ne disposons pas d'aimant permanent. Nous avons du créer un électroaimant à partir d'un
matériau ferromagnétique qui nous a été fourni. Cette section présente dans un premier temps le dimensionnement de l'électroaimant, c'est à dire le nombre 
de spires choisies, la résistance totale de la bobine...
Le nombre de spires total de la bobine a été choisi de manière arbitraire, sur les conseils de notre tuteur à 420. Pour ce nombre de spires, un entrefer 
réduit à 7mm et un courant de 1A, nous obtenons une champ magnétique de 75mT.
Avec 420 spires, nous obtenons une longueur de fil de 42.22m ce qui avec notre résistance linéique de 0.18 \Ohm /m nous donne une résistance totale de 7.6 \Ohm .
Ensuite la bobine mobile va intercepter le champ magnétique de la bobine fixe qui va lui permettre de faire bouger la membranne,
dans notre cas de 2mm. Pour obtenir cettre force il faut que la force magnétique soit égal à la force mécanique égal à kx, le nombre de spires nécessaire
pour obtenir cette force il nous faut un fil long de 9.57m donc 89.7 spires pour une résistance totale de 1.72 \Ohm
\begin{itemize}
    \item Fonctionnement
    \item Champ magnétique dans l'entrefer
    \item résistance totale dans la bobine
    \item Constante de raideur
    \item bobine mobile
    \item calcul du nombre de spires
    \item résistance totale
\subsection{Conclusion}
\begin{itemize}
    \item Nos erreurs
    \item Nos apprentissages
    \item Notre ressentit
\section{Antoine}
\subsection{Filtres}

\subsection{Boucles de contre-réaction}

\section{Marie-Charlotte}
\subsection{Fonctionnement général}

Le GSM ou le MP3 va dans un premier temps envoyer un signal audio via le câble Jack au circuit imprimé. Celui-ci permet
de modifier ce signal brut de plusieurs façons :

\begin{itemize}
	\item En réglant le volume, c'est-à-dire en modifiant l'amplitude du signal audio ;
	\item En réglant les graves et les aigus grâce au filtre passe-bande ;
	\item En amplifiant le signal : c'est le rôle de l'amplificateur audio du circuit.
\end{itemize}

À la sortie du circuit imprimé, le signal est alors \textit{filtré} et \textit{amplifié}.
Ce signal traité ira ensuite alimenter en courant la bobine mobile. Cette dernière intercepte un champ magnétique constant 
produit par un électroaimant formé d'une bobine de fil de cuivre, d'un matériau ferromagnétique, et alimenté en courant continu.
La bobine mobile subit donc une force de \textsc{Laplace}, va osciller en fonction du  signal audio, et reproduira le son voulu.
Enfin, la membrane pourra revenir à sa position d'équilibregrâce à des attaches qui, à la manière de ressorts, produisent une
force de rappel dans la direction opposée au mouvement de la bobine mobile. 

\subsection{Dimensionnement}
\subsubsection{Le boîtier}
La première question qui s'est posée était celle du volume du caisson. Nous nous sommes renseignés sur la fabrication de 
haut-parleurs, et nous avonsappris que le volume est lié à de nombreux paramètres, et qu'un volume trop petit ne restituerait 
pas bien les extrèmes graves. Etant donné que nous devions pouvoir faire varier la fréquence sur notre dispositif, nous avons 
finalement opté pour un boîtier cubique de $\unit{25}{\centi\meter}$ de côté.Afin d'améliorer un peu le boîtier, nous avons
également pensé à placer des pieds en caoutchouc afin de réduire les déplacements dûs aux vibrations du haut-parleur.Enfin,
pour faciliter l'accès à l'intérieur du haut-parleur, nous avons créé un système de porte coulissante à l'arrière de celui-ci.

\subsubsection{La membrane}
Nous avons opté pour une membrane de diamètre de $\unit{17}{\centi\meter}$, pour exploiter le mieux possible la taille du
caisson. C'est également undiamètre assez répandu dans le commerce. La profondeur correspondante est de $\unit{6}{\centi\meter}$.
Elle est réalisée en papier, et nous avons opté pour du tissus tendu en guise de ressort. Cette solution nous apparaît comme
sortant de l'ordinaire, propre, et efficace. Cela nous a en effet permis d'obtenir une constante de raideur minime.

\subsubsection{Modélisation du haut-parleur}
Pour écrire les équations du mouvement, nous avons d'abord fait l'inventaire des forces agissant sur le système. Ensuite, nous
avons négligé les frottements, et le poids de la bobine. Nous sommes finalement arrivés à l'équation reprise dans le slide. 
Nous l'avons réécrite sous forme d'équation différentielle pour ensuite résoudre cette dernière. Nous avons alors remarqué 
qu'à une certaine fréquence, le dénominateur était nul. C'est bien entendu une situation impossible, et nous avons alors
étudié cette situaton. La fréquence correspondante est en fait la fréquence de résonance; fréquence à laquelle le dispositif a
une réponse maximale (ok, pas très bien dit, je ferai encore deux trois recherches là dessus pour avoir des mots plus justes)

\end{document}
