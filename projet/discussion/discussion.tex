\input{../head.tex}

\section{Discussion}

% Théorie et pratique
En théorie, un son devrait sortir du haut-parleur.  En pratique, il y a bien un signal
à la sortie de la plaquette mais le haut-parleur demeure aphone. 

% Mesure et modélisation
Les mesures réalisées sont moins précises que la simulation, mais nous obtenons tout de même 
des valeurs assez proches de celle prédites.  Par exemple, nous obtenons un champ magnétique
expérimental de \unit{8}{\centi\tesla} alors que nos calculs nous prédisais un champ magnétique de
\unit{7.54}{\centi\tesla}. Ces différences s'expliquent par les hypothèses simplificatrices
que nous avons prises. Par exemple, nous avons supposé pour nos calculs que tout le champ se trouvait 
concentré dans l'entrefer de l'électroaimant.

% Système complet
Les points forts de notre système sont les suivants : la membrane, le caisson, et l'entrefer réduit de l'électroaimant. 
La membrane a été réalisée avec du tissus et du papier, et ainsi éviter les pliages difficiles, mais tout de même garder 
une certaine rigidité. Nous somme églament assez fier de notre caisson : robuste, avec un système d'ouverture par une porte
coulisante à l'arrière, équipé de pied en caoutchouc, il a tout d'un véritable caisson de haut-parleur. Enfin, nous avons 
pensé à réduire l'entrefer, afin d'augmenter le champ magnétique produit par l'électroaimant. Le principal point faible de 
l'appareil est qu'il n'y a pas de son sortant du haut-parleur.

% Gestion du travail de groupe
Nous avons assez bien géré le groupe, l'ambiance dans le groupe était excellente. Le travail était assez bien réparti
même si ce n'était pas possible que tout le monde ait la même charge de travail. Grâce à des outils comme GitHub, Dropbox et 
LaTeX, la communication pour la rédaction du rapport à été grandement facilitée.

\input{../foot.tex}
